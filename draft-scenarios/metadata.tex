% Set page size and margins
\usepackage[
  a4paper,
  top=2.43cm,
  bottom=3cm,
  left=1.5cm,
  right=1.5cm,
  marginparwidth=1.75cm,
  footskip=2.05cm,
]{geometry}

% General language packages
\usepackage[T1]{fontenc}
\usepackage{fontspec}

% Useful packages
\usepackage[export]{adjustbox}
\usepackage{amsmath}
\usepackage{array}
\usepackage{caption}
\usepackage[strict]{changepage}
\usepackage{enumitem}
\usepackage{etoolbox}
\usepackage{float}
\usepackage{fullwidth}
\usepackage{graphicx, trimclip}
\usepackage[colorlinks=true, allcolors=blue]{hyperref}
\usepackage{hyperref}
\usepackage[noautomatic, nonewpage]{imakeidx}
\usepackage{multicol}
\usepackage[super]{nth}
\usepackage{outlines}
\usepackage{paracol}
\usepackage[section]{placeins}
\usepackage{setspace}
\usepackage{stfloats}
\usepackage{subcaption}
\usepackage[usetransparent=false]{svg}
\usepackage{tabularx}
\usepackage[subfigure]{tocloft}
\usepackage{tikz}
\usepackage{titlesec}
\usepackage{transparent}
\usepackage{verbatim}
\usepackage{varwidth}
\usepackage{wrapfig}
\usepackage[most]{tcolorbox}
\usepackage{catchfile}
\usepackage{xstring}
\usepackage{soul}
\usepackage{xifthen}
\usepackage{xparse}

\usepackage{tocloft}
\renewcommand{\cftsubsecpagefont}{\bfseries}

\usepackage[hang, symbol, perpage]{footmisc}
\renewcommand{\footnotemargin}{1em}

\newtcolorbox{scaledfigure}[1][]{height fill, space to=\myspace,#1}
\hypersetup{
  colorlinks=true,
  linkcolor=goldenbrown,
  filecolor=magenta,
  urlcolor=cyan,
  pdftitle={Heroes of Might \& Magic III Fan-Made Draft Scenarios},
  pdfpagemode=UseNone,
}
% Set the default spacing between paragraphs. Remove indentation.
\usepackage[skip=6pt, indent=0pt]{parskip}
\setstretch{1}

% Default margins for itemize lists
\setlist[itemize,2]{leftmargin=15pt, label=$\triangleright$}
\setlist[enumerate,2]{leftmargin=15pt}

% Get version from env
% \getenv{variable_name} just prints the value
% \getenv[\macro]{variable_name} stores the value in \macro for reusability
\newcommand{\getenv}[2][]{%
  \CatchFileEdef{\value}{"|echo \$#2"}{\endlinechar=-1}%
  \if\relax\detokenize{#1}\relax\value\else\let#1\value\fi}

% Add dots to the table of contents
\renewcommand{\cftsecleader}{\cftdotfill{\cftsecdotsep}}
\renewcommand\cftsecdotsep{\cftdot}
\renewcommand\cftsubsecdotsep{\cftdot}

\captionsetup[figure]{labelformat=empty}
\captionsetup[subfigure]{labelformat=empty, singlelinecheck=off, justification=centering}
\usetikzlibrary{shadows, shadows.blur, calc, backgrounds}

\setlength{\columnsep}{1cm}
\newtoggle{printable}
\newtoggle{noartbackground}
\newtoggle{githubbuild}

% Variables
\def\_assets{assets}

\def\art{\_assets/art}
\def\cards{\_assets/cards}
\def\examples{\_assets/examples}
\def\images{\_assets/images}
\def\layout{\_assets/layout}
\def\maps{\_assets/maps}
\def\skills{\_assets/skills}
\def\spells{\_assets/spells}
\def\svgs{\_assets/glyphs}
\def\notes_svgs{\svgs/for-notes}
\def\tables{\_assets/tables}
\def\qr{\_assets/qr-codes}

\def\repourl{https://github.com/qwrtln/Homm3BG-mission-book}

\newcommand{\svg}[2][10]{%
  {\raisebox{-0.15\height}{\includesvg[height=#1px]{\svgs/\detokenize{#2}.svg}}}%
}%

\newcommand{\svgunit}[2][10]{%
  {\raisebox{-0.1\height}{\includesvg[height=#1px]{\svgs/\detokenize{#2}.svg}}}%
}%
\newcommand{\svgeven}[2][10]{%
  \includesvg[height=#1px]{\svgs/\detokenize{#2}.svg}%
}%

\renewcommand{\labelitemi}{
  \begin{tikzpicture}
    \node (listdot) [circle, inner sep=-3] {\includegraphics[width=1em, valign=c]{\layout/listdot.png}};
  \end{tikzpicture}
}

% Colors
\definecolor{amber}{rgb}{1.0, 0.49, 0.0}
\definecolor{antiquewhite}{rgb}{0.98, 0.92, 0.84}
\definecolor{arylideyellow}{rgb}{0.96, 0.89, 0.58}
\definecolor{bistre}{rgb}{0.24, 0.17, 0.12}
\definecolor{cadmiumgreen}{rgb}{0.0, 0.42, 0.24}
\definecolor{camel}{rgb}{0.76, 0.6, 0.42}
\definecolor{darkcandyapplered}{rgb}{0.64, 0.0, 0.0}
\definecolor{cobalt}{rgb}{0.0, 0.28, 0.67}
\definecolor{goldenbrown}{rgb}{0.6, 0.4, 0.08}

% Command to frame images
\newcommand\framedimage[2][]{%
  \begin{tikzpicture}
    \draw (0, 0) node[inner sep=0] {\makebox[#1][c]{\includegraphics[width=#1]{#2}}};
    \draw [bordermidyellow, thick] ([xshift=+1pt, yshift=-1pt] current bounding box.north west) rectangle ([xshift=-1pt, yshift=1pt] current bounding box.south east);
    \draw [borderoutyellow, thick] (current bounding box.north west) rectangle (current bounding box.south east);
    \draw [borderinyellow, thick] ([xshift=+3pt, yshift=-3pt] current bounding box.north west) rectangle ([xshift=-3pt, yshift=3pt] current bounding box.south east);
  \end{tikzpicture}}
% End of drop frame definition

\titleformat{\section}
{\huge}
{\filright
\footnotesize
\enspace SECTION \thesection\enspace}
{8pt}
{\Huge\bfseries\filcenter\uppercase}

\newfontfamily{\liberation}{LiberationSerif}
[
  Path = ../assets/fonts/,
  Extension = .ttf,
  UprightFont = *-Regular,
  ItalicFont = *-Italic,
  BoldFont = *-Bold,
  BoldItalicFont = *-BoldItalic
]

\newcommand{\sectionheadertext}[2][antiquewhite]{
  \color{#1}\MakeUppercase{\textbf{\liberation #2}}
}

%Create section heading with graphics. Argument one is heading name, argument two is picture to use on the left.
\newcommand{\addsection}[2]{
  \vspace*{-5.72em}
  \hspace*{-1.3em}
  \makebox[0pt][l]{
  \raisebox{-\totalheight}[0pt][7pt]{
      \begin{tikzpicture}
        \draw (0, 0) node[inner sep=0] (header){\makebox[1.015\textwidth][c]{\includegraphics[width=1.055\linewidth, height=0.24\linewidth]{\layout/section_heading.png}}};
        \draw (-6.7, 0) node {\includegraphics[width=0.135\textwidth]{#2}};
      \end{tikzpicture}
    }
  }
  \begin{fullwidth}[leftmargin=0.21\textwidth]
    \begin{center}
      \vspace{1em}
      \vspace{\lang_header_adjustment}
      \section*{\sectionheadertext{#1}}
      \cleardoublepage\phantomsection\addcontentsline{toc}{section}{\protect\numberline{}#1}
      \pagetarget{#1}{}
    \end{center}
  \end{fullwidth}
  \vspace{1.75em}
  \vspace{\lang_header_adjustment}
}
%End of create section heading.

% Add title page for Scenario type
\newcommand{\addscenariogroup}[2]{
  \thispagestyle{empty}
  \cleardoublepage\phantomsection\addcontentsline{toc}{section}{\protect\numberline{}#1}
  \AddToHookNext{shipout/background}{%
    \put (0in,-\paperheight){\includegraphics[width=\paperwidth,height=\paperheight]{\layout/tausta.png}}%
  }
  \begin{tikzpicture}[remember picture, overlay, inner sep=10pt]
    \node(cover)[anchor=center] at (current page.center) {
      \includegraphics[height=\paperheight, keepaspectratio]{#2}
    };
    \node(heading)[anchor=center] at (current page.center) {
      \includegraphics[width=\linewidth, keepaspectratio]{\layout/grouping_heading.png}
    };
    \node(title)[minimum width = \paperwidth, anchor=center] at (current page.center) {
      \huge\sectionheadertext[bistre]{#1}
    };
  \end{tikzpicture}
}

\newcommand\addheadershadow[2][]{
    % #1: Optional aditional tikz options
    % #2: Name of the node to "decorate"
    \begin{pgfonlayer}{background}
        \path[
           rounded corners=1pt,
           blur shadow={shadow xshift=0pt,
           shadow yshift=0pt,
           shadow blur steps=10,
           shadow blur radius=6pt}, #1]
            ($(#2.north west)+( 0.6pt,0)$) --
            ($(#2.south west)+( 0.6pt,0)$) --
            ($(#2.south east)+(-0.6pt,0)$) --
            ($(#2.north east)+(-0.6pt,0)$) --
        cycle;
        \path[rounded corners,
           blur shadow={shadow xshift=0pt,
           shadow yshift=0pt,
           shadow blur steps=10,
           shadow blur radius=6pt}, #1]
            ($(#2.north west)+(-1.3pt,-3pt)$) --
            ($(#2.south west)+(-1.3pt, 3pt)$) --
            ($(#2.south east)+( 1.3pt, 3pt)$) --
            ($(#2.north east)+( 1.3pt,-3pt)$) --
            cycle;
    \end{pgfonlayer}
}

% Four mandatory params
% - [optional] set to "subsection" or any other Level if you want to have this as a subsection in TOC
% - Lines of Campaign Name (1-2 according to the name length)
% - Campaign Name
% - Scenario Name
% - Icon
%
% TODO: possibly replace the whole \addsection with this
\newcommand{\addscenariosection}[5][section]{
  \sodef\sotitle{}{0.2em}{0.6em}{1.2em}
  \def\oneliner{\equal{#2}{1}}

  \vspace*{-5.72em}
  \hspace*{-1.3em}
  \makebox[0pt][l]{
  \raisebox{-\totalheight}[0pt][7pt]{
    \ifthenelse{\oneliner}{\def\yscale{1}}{\def\yscale{1.3}}
      \begin{tikzpicture}
        \draw (0, 0) node[inner sep=0, yscale=\yscale] (header){\makebox[1.015\textwidth][c]{\includegraphics[width=1.055\linewidth, height=0.24\linewidth]{\layout/section_heading.png}}};
        \draw (-6.7, 0) node {\includegraphics[width=0.135\textwidth]{#5}};
      \end{tikzpicture}
    }
  }
  \begin{fullwidth}[leftmargin=0.21\textwidth]
    \begin{center}
      \vspace{\lang_header_adjustment}
      \vspace{-12pt}
      \section*{\sectionheadertext{\small{\sotitle{#3}}}}
      \vspace{\lang_header_adjustment}
      \vspace{-10pt}
      \section*{\sectionheadertext{#4}}
      \ifthenelse{\oneliner}{}{\vspace{14pt}}
      \cleardoublepage\phantomsection\addcontentsline{toc}{#1}{\protect\numberline{} {} {} {} {}#4}
      \pagetarget{#4}{}
    \end{center}
  \end{fullwidth}
  \ifthenelse{\oneliner}{\vspace{1.75em}}{\vspace{0.75em}}
  \vspace{\lang_header_adjustment}
}

% Apply language-specific subsection spacings if defined
\ifdefined\subsectionspacing
  \subsectionspacing{}
\fi

\newcommand\picdims[4][]{%
  \setbox0=\hbox{\includegraphics[#1]{#4}}%
  \clipbox{.5\dimexpr\wd0-#2\relax{} %
    .5\dimexpr\ht0-#3\relax{} %
    .5\dimexpr\wd0-#2\relax{} %
    .5\dimexpr\ht0-#3\relax}{\includegraphics[#1]{#4}}}

\tikzset{
  thick/.style=      {line width=1.3pt},
  very thick/.style= {line width=1.7pt},
  ultra thick/.style={line width=2.2pt}
}

\definecolor{borderoutyellow}{HTML}{DBCA86}
\definecolor{borderinyellow}{HTML}{B09E69}
\definecolor{bordermidyellow}{HTML}{6f6749}
% Create note box
\newcommand{\notefont}[0]{\liberation\selectfont}
\newcommand{\note}[2]{
  \begin{tikzpicture}
    \draw (0, 0) node[inner sep=0] {\makebox[\linewidth][c]{\picdims[width=\linewidth]{\linewidth}{#1\baselineskip}{\layout/table-background.jpg}}};
    \draw [borderoutyellow, very thick] (current bounding box.north west) rectangle (current bounding box.south east);
    \draw [borderinyellow, thick] ([xshift=+2.8pt, yshift=-2.8pt] current bounding box.north west) rectangle ([xshift=-2.8pt, yshift=2.8pt] current bounding box.south east);
    \node at (current bounding box.center) {
      \begin{varwidth}{0.85\linewidth}
      \notefont{
        \color{arylideyellow}
        \hypersetup{linkcolor=amber}
        #2
        \hypersetup{linkcolor=goldenbrown}
      }
      \end{varwidth}
    };
    \begin{pgfonlayer}{background}
      \begin{scope}[blend mode=multiply]
        \draw [shade, blur shadow={shadow opacity=15}] (current bounding box.north west) rectangle (current bounding box.south east);
      \end{scope}
    \end{pgfonlayer}
  \end{tikzpicture}
}

% Create Heroes-styled framed canvas for a table. Accepts three arguments:
% 1) [Optional] Drop shadow description. Use [] as the first arg to delete it.
% 2) Height specified in verses (lines of text)
% 3) Table contents like title and tabularx environment
\newcommand{\hommtable}[3][shade, blur shadow={shadow opacity=15}]{
  \begin{tikzpicture}
    \draw (0, 0) node[inner sep=0] {\makebox[\linewidth][c]{\picdims[width=\linewidth]{\linewidth}{#2\baselineskip}{\layout/table-background.jpg}}};
    \draw [bordermidyellow, thick] ([xshift=+1pt, yshift=-1pt] current bounding box.north west) rectangle ([xshift=-1pt, yshift=1pt] current bounding box.south east);
    \draw [borderoutyellow, thick] (current bounding box.north west) rectangle (current bounding box.south east);
    \draw [borderinyellow, thick] ([xshift=+3pt, yshift=-3pt] current bounding box.north west) rectangle ([xshift=-3pt, yshift=3pt] current bounding box.south east);
    \node at (current bounding box.center) {
      \begin{varwidth}{0.95\linewidth}
      \notefont{
        \bgroup
        \color{arylideyellow}
        \hypersetup{linkcolor=amber}
        \setlength{\tabcolsep}{0.3em}
        #3
        \egroup
      }
      \end{varwidth}
    };
    \begin{pgfonlayer}{background}
      \begin{scope}[blend mode=multiply]
        \draw [#1] (current bounding box.north west) rectangle (current bounding box.south east);
      \end{scope}
    \end{pgfonlayer}
  \end{tikzpicture}
}
% End of Heroes-styled canvas definition.

\definecolor{darkcellborder}{HTML}{634831}
\definecolor{darkcellbg}{HTML}{20160C}

\newcommand{\darkcell}[2][0.9]{
  \begin{tikzpicture}
    \filldraw[line width=1.0pt, fill=darkcellbg, fill opacity=0.5, draw=darkcellborder] (0, 0) rectangle (\linewidth, #1);
    \node[text width=\linewidth, align=center] at (current bounding box.center) {\textbf{#2}};
  \end{tikzpicture}
}

\definecolor{lightcellborder}{HTML}{77543e}
\definecolor{lightcellbg}{HTML}{20160C}

\newcommand{\lightcell}[2][0.9]{
  \begin{tikzpicture}
    \filldraw[line width=1.0pt, fill=lightcellbg, fill opacity=0.25, draw=lightcellborder] (0, 0) rectangle (\linewidth, #1);
    \node[text width=\linewidth, align=center] at (current bounding box.center) {\color{white}#2};
  \end{tikzpicture}
}

% Commands to be used for automation generating printable version
\newcommand{\pagetarget}[2]{\label{#1}\hypertarget{#1}{#2}}
\newcommand{\pagelink}[2]{\hyperlink{#1}{#2}\iftoggle{printable}{ \textmd{(\pageshorthand\,\pageref{#1})}}{}}

% Command for overlay circled text
\definecolor{goblin}{HTML}{3b7c33}
\newcommand\encircle[1]{%
  \tikz[baseline=(X.base)]
  \node (X) [draw=white, shape=circle, inner sep=0, fill=goblin, text=white, blur shadow={shadow blur steps=5}] {\strut \textbf{#1}};%
}

% Background
\AddToHook{shipout/background}{%
  \iftoggle{noartbackground}{}{
    \put (0in,-\paperheight){\includegraphics[width=\paperwidth,height=\paperheight]{\layout/tausta.png}}
  }
  \iftoggle{printable}{
    \ifodd\value{page}
      \put (0in,-\paperheight){\includegraphics[width=\paperwidth]{\layout/bottom-odd.png}}
    \else
      \put (0in,-\paperheight){\includegraphics[width=\paperwidth,height=0.05\paperheight]{\layout/bottom-even.png}}
    \fi
  }{\put (0in,-\paperheight){\includegraphics[width=\paperwidth,height=0.05\paperheight]{\layout/bottom.png}}}
}

\begin{document}

\include{sections/title_page.tex}

\iftoggle{printable}{
  \newgeometry{
    twoside,
    top=2cm,
    bottom=3cm,
    left=2.5cm,
    right=1.5cm,
    marginparwidth=1.75cm,
    footskip=2.05cm,
  }
}{}

\author{\authortext}
\maketitle

\begin{center}
  \iftoggle{githubbuild}{
    \getenv[\githubsha]{GITHUB_SHA}
    \versionwarning{} \href{\repourl}{\StrLeft{\githubsha}{7}}.
  }{
    \versionlabel{} \input{../.version}
  }

  \bigbreak

  \intro{}

  \bigbreak

  \includegraphics[width=0.4\linewidth]{\qr/github.png} \\
  \qrgithub
\end{center}

\AddToHookNext{shipout/background}{
  \put (0in,-\paperheight){\includegraphics[width=\paperwidth,height=\paperheight]{\layout/tausta.png}}
}
\thispagestyle{empty}
\begin{tikzpicture}[remember picture, overlay]
  \node(cover)[anchor=center, yshift=9em] at (current page.south) {
    \includegraphics[width=1.01\paperwidth, keepaspectratio]{\layout/rampart_background.png}
  };
\end{tikzpicture}

\clearpage

\begin{multicols*}{2}
\tableofcontents
\vspace*{\fill}
\columnbreak
\vspace{-3em}
\includegraphics[width=\linewidth]{\art/mummy.jpg}
\end{multicols*}

\clearpage

\addscenariogroup{\campaignstitle}{\layout/campaigns.png}

\clearpage

\input{\campaignspath/inferno_devilish_plan.tex}

\clearpage

% !TeX spellcheck = en_US
\addscenariosection[subsection]{1}{Inferno Campaign $-$ Dungeons and Devils}{2. Steadwick's Fall}{\images/firewall.png}

\begin{multicols*}{2}

\textbf{Author:} Tm335

\textbf{Source:} \href{https://discord.com/channels/740870068178649108/1246353361456861276/1246353361456861276}{Archon Studios Discord}

\textit{Catherine Ironfist has enlisted aid from Bracada and AvLee.
She knows we are close to Steadwick.
We must occupy Steadwick before she arrives.
Once we own Erathia's capitol, not even Catherine Ironfist will wrench it from our hands.}

\subsection*{\MakeUppercase{Scenario Length}}

This Scenario plays out over 16 Rounds.

\subsection*{\MakeUppercase{Player Setup}}

\textbf{Faction:} Inferno

\textbf{Faction Hero:} Choose any

\textbf{Starting Resources:} 15 \svg{gold}, 1 \svg{building_materials}, 1 \svg{valuables}

\textbf{Starting Income:} 10 \svg{gold}, 2 \svg{building_materials}, 0 \svg{valuables}

\textbf{Starting Units:}

\begin{itemize}
  \item A Few Troglodytes
  \item A Few Evil Eyes
  \item A Pack of Familiars
\end{itemize}

\textbf{Town Buildings:} \svgunit{bronze} Dwelling, \svgunit{silver} Dwelling

\textbf{Bonus:} Choose one of the following options:
\begin{itemize}
  \item Add a Pack of Harpies to your hand.
  \item Add a Pack of Magogs to your hand.
  \item Add an Ammo Cart to your hand.
  \item Search (4) the Spell Deck.
\end{itemize}

\subsection*{\MakeUppercase{AI Hero Setup}}

\textbf{Faction:} Castle

\textbf{Enemies:} General Kendal, Charging Heroes

\textbf{General Kendal's Army:} A Pack of Archangels, a Pack of Champions, a Pack of Zealots, a Pack of Crusaders, a Pack of Griffins, a Balista War Machine

\textbf{General Kendal's Deck:} 5 × Might Card, 1 × Magic Card, 3 × Skill Card

\textbf{General Kendal's Spell Deck:} 2 × Haste Spell Card

\textbf{General Kendal's Skill:} Artillery Ability Card\footnote{For General Kendal, the Artilery Ability Card always resolves the Expert effect and the Ballista War Machine Activates every time the Artillery Ability Card is drawn, as well as at the beginning of a Combat Round.}

\textbf{Charging Heroes' Factions:} Rampart, Tower, Castle

\textbf{Charging Heroes' Armies:} Neutral Army is one Level higher than your Hero Level (Max Level VI)\footnote{See page 35, ``Field Difficulty Level Table'' in the Core Rulebook, for further details on the number of Neutral Units you have to draw for this Neutral Army.}.

\textbf{Charging Heroes' Deck:} 2 × Might Card, 2 × Magic Card

\textbf{Charging Heroes' Spell Deck:} 2 × Slow Spell Card\footnote{All the Charging Heroes' Enemies use the same AI and Spell Decks. Reset them after every Combat.}.

\subsection*{\MakeUppercase{Map Setup}}

Take the following Map Tiles and set them up as shown in the Scenario map layout:

\textbf{2 × Starting (I) Map Tile}
\begin{itemize}
  \item 1 × Inferno (S6)
  \item 1 × Castle (S3)
\end{itemize}

\textbf{3 × Far (II--III) Map Tile}
\begin{itemize}
  \item 1 × Castle (F3)
  \item 1 × Rampart (F10)
  \item 1 × Dungeon (F2)
  \item 1 × Tower (\#F1)
  \item 1 × Rampart (choose from: F11, F12)
  \item 1 × Necropolis (choose from: F4, \#F6, F7)
\end{itemize}

\textbf{2 × Near (IV--V) Map Tile}
\begin{itemize}
  \item 2 × Castle (N3, choose from: \#N3, N6)
  \item 1 × Rampart (N8)
  \item 1 × Necropolis (N4)
\end{itemize}

\subsection*{\MakeUppercase{Heroes Placement}}

The Enemy Hero General Kendal is represented by one Castle Faction Hero model and appears on the center Field of the S3 Starting Map Tile.

The Castle Charging Hero is represented by one Castle Faction Hero model and appears on (and owns) the Settlement of the F3 Map Tile.

The Rampart Charging Hero is represented by one Rampart Faction Hero model and appears on (and owns) the Settlement of the F10 Map Tile.

The Tower Charging Hero is represented by one Tower Faction Hero model and appears on (and owns) the Settlement of the \#F1 Map Tile.

Place your Main Hero on the center Field of the Inferno Starting S6 Map Tile.

Place your Secondary Hero, represented by one Dungeon Faction Hero model of your choice, on the Settlement of the F2 Map Tile. This Settlement produces no resources.

\subsection*{\MakeUppercase{Victory Conditions}}

Defeat all 3 Enemy Controlled Settlements (F3, F10, \#F1), and capture Steadwick (S3) before
Queen Catherine Ironfist arrives at the end of the Round 16.

\subsection*{\MakeUppercase{Defeat Conditions}}

You lose one Combat encounter with your Main Hero (Surrendering costs 10 \svg{gold}, and does not count as a defeat).

You lose your Faction Town on the S6 Map Tile.

You run out of time -- you have time till the end of the Round 16.

\subsection*{\MakeUppercase{Timed Events}}

\textbf{\nth{1} Round:}
\begin{itemize}
  \item Read: ``You are to be congratulated on your progress so far.
    You have laid waste to Eastern Erathia, and are now within striking distance of the Erathian
    capital of Steadwick. You must capture the capital quickly!''
\end{itemize}

\textbf{\nth{2} Round:}
\begin{itemize}
  \item Read: ``Not only have Bracada and AvLee sent reinforcements, but we have received news that
    Queen Catherine Ironfist is marching a sizeable army from the south. We must control the capital and its
    garrisons before she arrives.

    You have just received a report on the progress of Queen Catherine.
    Forces from Nighon and Eeofol are attempting to delay her march to Steadwick,
    but doubt that they can delay her more than two or three months.''
\end{itemize}

\textbf{\nth{5} Round:}
\begin{itemize}
  \item Read: ``You receive a report from the south. Queen Catherine's forces have been sufficiently delayed,
    allowing you at least two months more to reach the capitol, but our own forces have suffered significant
    losses. Do not let their sacrifice go to waste.''
\end{itemize}

\textbf{\nth{10} Round:}
\begin{itemize}
  \item For any Enemy Settlements (F3, F10, \#F1) that have not been defeated, an additional Enemy
    Charging Hero of that Faction appears on each Settlement at the beginning of this Round.
\end{itemize}

\textbf{\nth{11} Round:}
\begin{itemize}
  \item Read: ``You receive a report from the south. Our forces continue to throw themselves in the path of
    Queen Catherine's armies, yet she continues to march northward. You have, at most, three or four weeks
    before she can reach the capital.''
\end{itemize}

\textbf{\nth{13} Round:}
\begin{itemize}
  \item Warning: ``Queen Catherine's march continues -- her forces are just two weeks away. If you do not hurry,
    we will not have time to secure the capital before her arrival.''
\end{itemize}

\textbf{\nth{15} Round:}
\begin{itemize}
  \item Warning: ``Your scouts report sighting Queen Catherine's army seven days to the southwest. If she
    reaches the capitol before you, all is lost.''
\end{itemize}

\textbf{\nth{16} Round:}
\begin{itemize}
  \item If Steadwick has not been taken, read: ``This morning, a massive army lead by Queen Catherine
    Ironfist arrived at the Erathian capitol of Steadwick. We have no choice but to retreat our forces.
    You have failed us... miserably.''
\end{itemize}

\textbf{When you complete the Scenario:}
\begin{itemize}
  \item Read: ``Congratulations! You captured Steadwick and are victorious!''
\end{itemize}

\subsection*{\MakeUppercase{Additional Rules}}

During this ``Inferno'' Campaign Scenario, the following rules apply:

\begin{itemize}
    \item The General Kendal Hero does not move and only waits in their Town. They start with Walls, a Gate,
      and an Arrow Tower on their side of the Combat Board.
    \item The difficulty Level of every Combat encounter on the map increases by one till the end of the Scenario
      (see page 35, ``Field Difficulty Level Table'' in the Core Rulebook).
    \item The Enemy Charging Heroes have only 2 Movement Points, instead of 3. They ignore everything else
      (including Mines and Settlements if possible) and go straight for the player's Faction Town (on Map Tile S6).
      They do not pursue the player directly, but if they happen to be on the same Map Tile, they will
      attack the player's Main Hero (not Secondary).
    \item The Enemy Charging Heroes move after the player's Turn ends.
    \item The borders on the Castle Starting Map Tile S3 cannot be crossed by any means until
      all Enemy Settlements (F3, F10, \#F1) have been defeated.
    \item Defeating Enemy Charging Heroes provides 2 \svg{valuables}.
    \item Obelisks give you 1 \svg{valuables} and function as a ``Castle
      Gate'' location (only if you have the ``Castle Gate'' built in your Town).
\end{itemize}

\vspace*{\fill}
\begin{center}
  \framedimage[0.9\linewidth]{\art/demon.jpg}
\end{center}
\vspace*{\fill}

\end{multicols*}

\begin{tikzpicture}[remember picture, overlay]
  \node(bg)[anchor=center, yshift=37em, opacity=0.17] at (current page.south) {
    \includegraphics[width=0.85\paperwidth, keepaspectratio]{\art/fire_shield.png}
  };
  \node(map)[anchor=center] at (current page.center) {
    \includegraphics[width=\textwidth]{\maps/inferno_steadwicks_fall.png}
  };

  \node at (3, -11.5) {\large{{\textbf{\textcolor{darkcandyapplered}{N3}}}}};
  \node at (4.4, -18.6) {\large{{\textbf{\textcolor{darkcandyapplered}{\#F1}}}}};
  \node at (15, -5.2) {\large{{\textbf{\textcolor{darkcandyapplered}{F10}}}}};
  \node at (17.3, -11.5) {\large{{\textbf{\textcolor{darkcandyapplered}{S6}}}}};
  \node at (8.8, -18.8) {\large{{\textbf{\textcolor{darkcandyapplered}{F3}}}}};
  \node at (15.9, -18.8) {\large{{\textbf{\textcolor{darkcandyapplered}{F2}}}}};
  \node at (2, -4.5) {\large{{\textbf{\textcolor{darkcandyapplered}{Steadwick}}}}};
  \node at (2, -5.2) {\large{{\textbf{\textcolor{darkcandyapplered}{S3}}}}};
  \node at (7.5, -5.4) {\large{{\textbf{\textcolor{darkcandyapplered}{Rampart}}}}};
  \node at (7.5, -6.1) {\large{{\textbf{\textcolor{darkcandyapplered}{Map Tiles}}}}};
  \node at (12.4, -17.8) {\large{{\textbf{\textcolor{darkcandyapplered}{Necropolis}}}}};
  \node at (12.4, -18.5) {\large{{\textbf{\textcolor{darkcandyapplered}{Map Tiles}}}}};
  \node at (2, -12.3) {\large{{\textbf{\textcolor{darkcandyapplered}{Castle}}}}};
  \node at (2, -13) {\large{{\textbf{\textcolor{darkcandyapplered}{Map Tiles}}}}};
\end{tikzpicture}


\clearpage

\input{\campaignspath/inferno_deal_with_the_devil.tex}

\clearpage

% !TeX spellcheck = en_US
\cleardoublepage\phantomsection\addcontentsline{toc}{section}{\protect\numberline{} {} {} {} {}The Queen's Gambit}
\addscenariosection[subsection]{1}{Castle Campaign $-$ The Queen's Gambit}{1. Greek Gift}{\images/catherine.png}

\begin{multicols*}{2}

\textbf{Author:} Tyson Heckert

\textbf{Source:} \href{https://travelledtales.com}{travelledtales.com}

\textit{Steadwick is free, but that doesn't mean it is safe.
Dark powers still roam Erathia, and one now moves its pieces into position.
While Catherine grapples with her responsibility — and her demons — she must be a bulwark against evil.}

\subsection*{\MakeUppercase{Scenario length}}

This Scenario plays out over 8 Rounds.

\subsection*{\MakeUppercase{Player setup}}

\textbf{Faction:} Castle

\textbf{Faction Hero:} Any Castle Hero. If you are not Catherine, you are in her entourage for this story.

\textbf{Starting Resources:} 10 \svg{gold}, 3 \svg{building_materials}, 1 \svg{valuables}

\textbf{Starting Income:} 10 \svg{gold}, 0 \svg{building_materials}, 0 \svg{valuables}

\textbf{Starting Units:}
\begin{itemize}
  \item A Few Halberdiers, a Few Marksmen
\end{itemize}

\textbf{Town Buildings:} \svgunit{bronze} Dwelling, Citadel

\subsection*{\MakeUppercase{AI Hero setup}}

\textbf{Faction:} Necropolis

\textbf{Enemy Army:} A Pack of Skeletons, a Pack of Zombies, a Few Dread Knights, a Few Ghost Dragons

\textbf{Enemy Deck:} 4 x Might Cards

\textbf{Special:} Place the Cloak of the Undead King IV and VI Cards aside

\subsection*{\MakeUppercase{Map setup}}

Take the following Map Tiles and arrange them as shown in the Scenario map layout:

\textbf{1 × Starting (I) Map Tile}
\begin{itemize}
    \item 1 x Castle (S3)
\end{itemize}

\textbf{3 × Far (II--III) Map Tile}
\begin{itemize}
    \item 3 x Castle (F3, F6, F9)
\end{itemize}

\textbf{2 × Near (IV--V) Map Tile}
\begin{itemize}
    \item 2 x Castle (N3, N6)
\end{itemize}

\subsection*{\MakeUppercase{Heroes placement}}

Place your Castle Hero on the center Field of the S3 Castle starting map Tile.

Place a Necropolis Hero on the center Field of the N3 Castle Tile.

\subsection*{\MakeUppercase{Victory Conditions}}

Defeat the enemy army.

\subsection*{\MakeUppercase{Defeat Conditions}}
\begin{itemize}
  \item You lose one combat encounter
  \item You run out of time at the end of the 8th Round
\end{itemize}
\end{multicols*}

\newpage

\begin{multicols}{2}

\subsection*{\MakeUppercase{Timed Events}}

\textbf{\nth{1} Round:}
\begin{itemize}
  \item Read ``Distant Threat'' section
\end{itemize}

\textbf{If entering the settlement on F3:}
\begin{itemize}
  \item Read ``Reverent Allies'' section
\end{itemize}

\textbf{Before combat with the enemy Hero:}
\begin{itemize}
  \item Read ``Bait and Switch'' section
\end{itemize}

\textbf{When you complete the Scenario:}
\begin{itemize}
  \item Read ``The Chase Begins'' section
\end{itemize}

\subsection*{\MakeUppercase{Additional rules}}

During this Castle campaign Scenario, the following rules apply:

\begin{itemize}
  \item The enemy army does not move
  \item Your Hero's maximum Level is 4
  \item The Glory of Erathia building is unavailable
  \item The Settlement on F3 does not have neutral enemies
\end{itemize}

\columnbreak

\includegraphics[width=\linewidth, keepaspectratio]{\art/griffin.png}

\end{multicols}

\begin{tikzpicture}[overlay]
  \node at (8.6, -5.3) {
    \includegraphics[width=\textwidth]{\_assets/maps/castle_greek_gift.png}
  };
  \node at (10, -2.0) {\large{{\textbf{\textcolor{darkcandyapplered}{F3}}}}};
\end{tikzpicture}

\newpage

\begin{multicols*}{2}

\subsection*{\MakeUppercase{The story}}

\textbf{Distant Threat}

Catherine Ironfist rubbed her tired, worn thumb across the image of her father, King Gryphonheart, in a locket.
She wanted to remember the good, valiant parts of his life rather than his actions in unlife.
Her gaze turned to her cracked skin and mangled thumbnail, and it reminded her of how hard she had fought to be where she was.

The weight of Erathia rested on her shoulders.
It had only been weeks since the siege and recapture of Steadwick, and resources, as well as manpower, were in short supply.
Still, with time, the stony seat of the castle would be sufficient to further liberate the realm.

One such problem was weighing on her mind.
After replacing the locket in her hand with a missive, she re-read a particular entry about a rising necromancer named Vhidhax.
Rumors were that the fiend had stolen precious artifacts that further enhanced his power over the dead.
While rumors are usually best left to fishwives and children, this one aligned with information about an army growing in strength, and she dreaded the thought of the two problems colliding and crashing over her weary crown.

Late that evening, Catherine was venting to Rion about the relative drudgery of rebuilding Steadwick.
She'd just about fallen asleep in her chair when a heavy knock turned her attention to the room's large door that sported intricately carved shields from various long-gone baronies.

The wooden portal flung open before she could spring up, revealing a wide-eyed and panicked guard.
No words came from the man; whatever haunted him seemed absolute.
Instead, he merely pointed with a trembling hand toward the window across the room.

Catherine shot a look at Rion before the two darted toward the open air of the rectangular window.
What they saw shocked them, too, and while Catherine's heart sank, she gripped Rion's shoulder with surprising strength.
When he looked back at her with a slight wince, she only returned a stony look of resolve.

Danger had come to Steadwick again.
Atop a hill in the far distance, overlooking many of the defenseless farmsteads Catherine had in her charge, a predator swept its gaze like a wolf might hungrily eye sheep.
A magnificently terrifying azure hue wafted from a Ghost Dragon's massive, extended wings like an ethereal heat haze.

\textbf{Reverent Allies}

While Catherine trusted her horse, having fought tooth and nail with the battle-hardened steed for as long as she could remember, it shuddered in hesitation as they approached the majestic, columned main building of the settlement.
Catherine had also felt the almost unsettling holy power in the area when, suddenly, the sound of rushing wings caught her attention.
Her gaze snapped upward to catch sight of something approaching.

The early morning sun was bright, and the figure that swooped low was in front of it.
The silhouette of wings was visible, and Catherine thought it to be a Gryphon at first.
However, as it closed in on them, it became apparent it was the shape of a man with wings instead.
It caught Catherine's horse off guard, and she had to catch the reins and wrench them hard to steady herself and calm the frightened animal before it could fling her from her seat.

With lightning speed, the figure landed mere feet from them to reveal the stubbled, handsome face of a young man with golden, curled locs and impressively robust bronze armor sculpted to his muscular form.
He was one of the famed Archangels to have stayed near Steadwick after Catherine's victory.
While the forces behind her fell to their knees in reverence, she only nodded with stern resolve.

With a deeper, and more reverberating voice than any man, the angel was first to speak.
``Queen Catherine.
I expected you might come when I spied the rot taking form in the hills.''

Catherine responded with a smile.
It was good to see she still had allies; her forces would need bolstering to take on the undead.
``It's good to see you, my friend,'' she said.
``It seems Steadwick is threatened again, and I ride to root out the evil before it can seed.
Would you join me at my side?''

``Of course, my queen!'' the angel barked.
``Combined, no evil can thwart our advance!'' As the final words left his lips and began to echo, he drew his weapon.
The longsword ignited as it left its sheath, and its blazing glory had most of those kneeling averting their eyes.
Immediately after, he took to the air and began circling the area again.
Catherine smiled wide as she watched more angels appear, then looked back to her forces and extended an open palmed arm forward in a command to march.

\textcolor{darkcandyapplered}{Add a Few Archangels to your army.}

\textbf{Bait And Switch}

Catherine couldn't wait to drive her sword through the heart of the Ghost Dragon, then lop the head from the one commanding it.
She pushed her powerful mount forward, charging toward the waiting, taunting foes.
While the great number of shambling undead, mostly animated citizens from nearby townships, was impossible to ignore, they wouldn't stop her.
With a thunderous crash, she slammed past the forward lines, knocking off limbs and cracking bones.

Seconds later, she was halfway through the mass of bodies when the stench of rotten flesh and the true immensity of her enemy's numbers hit her.
Dread Knights flanked the dragon she'd been racing towards.
Worse still, a robed, decrepit-looking figure was watching from the rear of the enemy lines.
Despite its wretched appearance, it had an immensely powerful presence to it.
A stinging tingle shot up Catherine's spine as she felt, more than saw, its eyes return her gaze.
More than she could take on alone, she swung her steed in a circle to reform with her mounted allies, then looked to the sky just as an Archangel came crashing down onto the dragon.

Shockingly, instead of slamming into the beast, the angel flew through it like it was made of air.
He hit the ground with a crunching thud before, suddenly, the dragon's image began to shift and shimmer, then disintegrate.
The Dread Knights did the same, and replacing them were throngs of further undead that now threatened to encircle Catherine and her forward forces.

Catherine's eyes shot toward what she believed was the necromancer.
It stood motionless a moment longer before cackling laughter cut through the air, and its image wafted away into the night air just as the others had.

It was a trap.
The necromancer and his strongest forces were not there, and Catherine would have to fight her way out from under a crush of undead, wasting precious time and lives.

\begin{itemize}
  \item \textcolor{darkcandyapplered}{Replace the Ghost Dragons and Dread Knights with \svgunit{bronze} neutral Zombies and Skeletons.}
  \item \textcolor{darkcandyapplered}{Play ``Cloak of the Undead King IV'' and ``Cloak of the Undead King VI'' onto the packs of Zombies and Skeletons, respectively.}
  \item \textcolor{darkcandyapplered}{Remove the AI Might and Magic Deck from the game, the undead fight on their own.}
\end{itemize}


\textbf{The Chase Begins}

Catherine huffed, sweat beading on her forehead as the last undead fell with wet thuds.
Her forces were strong, and simple reanimations weren't the kind of threat to stop them.
She just hoped… Her stomach churned as her gaze turned back to Steadwick.

An orange glow, far too bright for torchlight, was coming from the castle.
As she watched in horror, thick black smoke began to rise from the towers and parapets.
Steadwick was burning.

Rion quickly caught up to her, his horse panting as hard as he was.
He began to speak but couldn't find the words as his mind raced.
Instead, he placed a hand on his queen's shoulder.

The necromancer had played his opening move and won.

Just before Catherine turned to her regrouping forces, she spotted the spectral blue of the real Ghost Dragon in the distance.
Its work destroying Steadwick's defenses had been finished.

Archangels circled overhead, and Catherine threw her head to the sky, tears welling at the thought of losing what she had worked so hard to build.
``Find him!'' she roared, addressing whatever gods were listening as well as the angels above.

``Is that… wise?'' Rion asked softly.

Catherine responded with fire in her eyes.
``What good is rebuilding a ruin when the fiend will just continue to play his tricks?'' she said.
``No.
If we want to save Steadwick, we have to end him before he can further strengthen himself.''

\begin{itemize}
  \item \textcolor{darkcandyapplered}{Remove Archangels from your army.}
  \item \textcolor{darkcandyapplered}{Keep the rest of your forces and resources for the next Scenario.}
  \item \textcolor{darkcandyapplered}{Reset your Hero per the normal campaign rules.}
\end{itemize}

\columnbreak

\begin{center}
  \framedimage[0.8\linewidth]{\art/ghost_dragon.jpg}
\end{center}

\end{multicols*}


\clearpage

% !TeX spellcheck = en_US
\addscenariosection[subsection]{1}{Castle Campaign $-$ The Queen's Gambit}{2. Sicilian Dragon}{\images/catherine.png}

\begin{multicols*}{2}

\textbf{Author:} Tyson Heckert

\textbf{Source:} \href{https://travelledtales.com}{travelledtales.com}

\textit{Convinced that the only way to save Steadwick is to attack Vhidhax, Catherine powers her small force ahead.
Following the Archangels to a distant, inhospitable land, she must now navigate her way through.}

\subsection*{\MakeUppercase{Scenario length}}

This Scenario plays out over 11 Rounds.

\subsection*{\MakeUppercase{Player setup}}

\textbf{Faction:} Castle

\textbf{Faction Hero:} The same Hero as Scenario 1

\textbf{Starting Resources:} The resources you ended with in Scenario 1

\textbf{Starting Income:} 10 \svg{gold}, 0 \svg{building_materials}, 0 \svg{valuables}

\textbf{Starting Units:} The army you ended with in Scenario 1

\textbf{Town Buildings:} None

\textbf{Bonus:} Begin the Scenario at Hero Level 2, but do not take new Ability Cards.

\subsection*{\MakeUppercase{AI Hero setup}}

\textbf{Faction:} Necropolis

\textbf{Enemy Army:} A Pack of Skeletons, a Pack of Zombies, a Pack of Liches, a Few Dread Knights, a Few Ghost Dragons

\textbf{Enemy Deck:} 3 x Might Cards, 3 x Magic Cards, 1 x Skill Card

\textbf{Enemy Spells:} 2 x Lightning Bolt, 1 x Curse

\textbf{Enemy Skills:} Cloak of the Undead King IV and VI, the enemy will play whichever it can

\textbf{Neutral Enemy Deck:} 4 x Might Cards

\subsection*{\MakeUppercase{Map setup}}

Take the following Map Tiles and arrange them as shown in the Scenario map layout:

\textbf{1 × Starting (I) Map Tile}
\begin{itemize}
    \item 1 x Necropolis (S1)
\end{itemize}

\textbf{3 × Far (II--III) Map Tile}
\begin{itemize}
    \item 3 x random Tiles from Necropolis or Dungeon (F1, F2, F4, F5, F7, F8)
\end{itemize}

\textbf{4 × Near (IV--V) Map Tile}
\begin{itemize}
    \item 2 x Necropolis (N1, N4)
    \item 2 x Dungeon (N2, N5)
\end{itemize}

\textbf{1 × Center (VI--VII) Map Tile}
\begin{itemize}
  \item 1 × Grail Tile (C2)
\end{itemize}

\textbf{\MakeUppercase{Note:}} Place Tile S1 aside, face up.

\subsection*{\MakeUppercase{Heroes placement}}

Place your Castle Hero on the empty Field of the rightmost Far Tile.

Place a Necropolis Hero on the center Field of the S1 Necropolis Tile.

\subsection*{\MakeUppercase{Victory Conditions}}

\begin{itemize}
  \item Find the Archangels
  \item Defeat the enemy army
\end{itemize}


\subsection*{\MakeUppercase{Defeat Conditions}}
\begin{itemize}
  \item You lose one combat encounter
  \item You run out of time at the end of the 11th Round
\end{itemize}

\end{multicols*}

\begin{multicols}{2}

\subsection*{\MakeUppercase{Timed Events}}

\textbf{\nth{1} Round:}
\begin{itemize}
  \item Read ``Into the Badlands'' section
\end{itemize}

\textbf{Visiting the first Obelisk:}
\begin{itemize}
  \item Read ``The First Relic'' section
\end{itemize}

\textbf{Visiting the second Obelisk:}
\begin{itemize}
  \item Read ``Showdown'' section
\end{itemize}

\textbf{When you complete the Scenario:}
\begin{itemize}
  \item Read ``The Escape'' section
\end{itemize}

\subsection*{\MakeUppercase{Additional rules}}

During this Castle campaign Scenario, the following rules apply:

\begin{itemize}
  \item The enemy army does not move
  \item Your Hero's maximum Level is 6
  \item All buildings are unavailable, building is not possible
\end{itemize}

\columnbreak

\begin{itemize}
\item All neutral enemies fight using the Neutral Enemy Deck.
\item Each time you visit a Trading Post, draw the top Cards from the \svgunit{bronze} bronze, \svgunit{silver} silver, and \svgunit{golden} golden neutral enemy Decks.
You may recruit any, or all of them by paying their cost.
Shuffle any un-hired ones back into their Decks.

\item The Grail may be sold at a trading post for 30 \svgunit{gold} \textit{or} \textbf{Search (3)} the relic Deck, twice.

\item When fighting the enemy at thier Town, they have a Citadel.
Don't forget to add the walls and arrow tower to make the battle a siege.
\end{itemize}

\end{multicols}

\begin{tikzpicture}[overlay]
  \node at (9.0, -6.0) {
    \includegraphics[width=0.80\paperwidth]{\_assets/maps/castle_sicilian_dragon.png}
  };
\end{tikzpicture}

\newpage

\begin{multicols*}{2}

\subsection*{\MakeUppercase{The story}}

\textbf{Into the Badlands}

Catherine crossed arms and leaned on her saddle horn as she peered out over the wasteland they'd been led to.
The Archangels had gone ahead of the small force she'd scraped together, and she wondered if it had been a mistake to let them go.
With the enemy stronghold still nowhere in sight, locating it would take time, and the inhospitable landscape would not be kind.

Rion pulled up next to her as he finished gulping the last contents of a water pouch.
He wiped the dribble and sweat from his face before saying anything.
``Not the best place to do any fighting.
Besides the poor conditions, I've heard the locals are known for their particular… ferocity.''

``Well then, we'll just have to bring them to heel,'' Catherine replied coldly.
The truth was she was more worried about that possibility than she let on, but she had to maintain her composure to protect the image of her strength.

``You think they'll fight for us?'' Rion asked with a raised eyebrow.

Catherine tightly gripped a heavy sack of coin by her side and heaved it into her lap so the contents jingled like tiny bells to emphasize her point.
``I think generous arrangements can be made.
For the rest, the sword will do.''

\textbf{The First Relic}

Ahead, a clearing of dry, cracked ground gave way to a large Obelisk protruding from the ground like a massive marker.
Catherine had seen enough magical artifacts in her time that she could tell the structure was a place of power.
Throwing an arm forward, she signaled Rion to proceed with an inspection.

The mage approached slowly, removing a glove to touch the cold stone of the smoothly cut sides with his skin.
Intricate runes were carved in it, detailing all manner of scripture in long-lost tongues.
As awe-inspiring as it was, equally apparent was the time-worn damage.

After a moment, he turned to Catherine.
``It's a transport beacon,'' he said.
``Like a monolith, but with a shorter range.
At least… it was.
It's damaged, and where the magic in the runes should be strong, even glowing, they're dark, and any power is long decayed.''

Catherine dismounted and slowly inspected the area before saying anything.
``Some of the damage is recent.
I can still feel residual heat; the Archangels were here.''

``How can you be sure? Surely, some would have found us by now and notified us of their progress.''

``They've never let me down,'' Catherine grunted.
``Come, let's see if other such beacons exist.''

\textbf{Showdown}

Sounds of battle perked the ears of Catherine's forces, and she urged her steed forward to catch sight of the combatants.

Rounding crags and jagged rocks jutting at all angles, another Obelisk revealed itself with the final moments of a battle taking place around it.
Several Archangels were on the ground, locked in combat with the remains of a larger undead force.

Approaching at speed, Catherine raised a fist in the air to salute the holy warriors, and several rasped their bronze chestplates in return.

``Queen Catherine, you've saved us the trouble of finding you,'' the lead angel said in his deep voice.
He gestured to the Obelisk with an open palm as the sounds of battle finally died.
``The foes meant to keep us from these structures by destroying them.
We barely managed to intercept them in time to preserve this one.''

Rion trotted past them as they spoke, intent on inspecting the intact Obelisk.
As before, he began to place his hand on the cold stone, but recoiled before making contact.
``It's active!'' he said excitedly.
``This may be the break we need.
With luck, this will take us directly to the enemy stronghold.''

Catherine smiled, reaching to clasp the arm of the angel in thanks for his work.
She looked back at the ragtag group of forces she'd gathered and hoped it would be enough for a siege.

\begin{itemize}
  \item \textcolor{darkcandyapplered}{Add a Few Archangels to your army.}
  \item \textcolor{darkcandyapplered}{You may now teleport to the enemy town on Tile S1.
  If you choose not to, you may in the future by spending one movement at this Obelisk.}
\end{itemize}


\textbf{The Escape}

With the thick, macabre, almost demonic-looking walls breached and the bulk of the necromancer's forces slain, the castle belonged to Catherine.
It was an eye for an eye, and she relished driving the final blow into the last enemy ranks still in the courtyard herself.

Unfortunately, after a frantic search of the grounds, Vhidhax himself was nowhere in sight.
Catherine raced up the walls just in time to peer over the side and catch sight of the necromancer fleeing with a sizable retinue that had quietly slipped away during the battle.
He was traveling north, further into the badlands and away from her home.
Catherine's campaign wasn't finished yet.

Catherine turned to observe the battlefield from above.
The butcher's bill had been large.
Both fresh and long-decaying bodies lay strewn about, several Archangels among them.
Rion caught up with her a minute later, and the two exchanged looks of exhaustion.

``It's done,'' Rion said through panting breaths.

``No.
The necromancer lives, and he flees north.''

``But… our forces.
The Archangels are too few to continue; what hope do we have?''

``There is always hope, my friend.
We'll use this castle as a base to resupply, and then we need to give chase before the fiend regains his strength.
Steadwick will never be safe otherwise!''

``This is foolish,'' Rion sighed.
``We don't need to fight this hard; Vhidhax is not your father, Catherine…''

\begin{itemize}
  \item \textcolor{darkcandyapplered}{Remove Archangels from your army, if possible.}
  \item \textcolor{darkcandyapplered}{Keep the rest of your army and resources for the next Scenario.}
  \item \textcolor{darkcandyapplered}{Reset your Hero per the normal campaign rules.}
\end{itemize}

\vspace{2em}
\begin{center}
  \framedimage[0.75\linewidth]{\art/necropolis_town.jpg}
\end{center}

\end{multicols*}


\clearpage

% !TeX spellcheck = en_US
\addscenariosection[subsection]{1}{Castle Campaign $-$ The Queen's Gambit}{3. Two Knights Defense}{\images/catherine.png}

\begin{multicols*}{2}

\textbf{Author:} Tyson Heckert

\textbf{Source:} \href{https://travelledtales.com}{travelledtales.com}

\textit{With Vhidhax's stronghold in her grasp, Catherine needs to discover and halt the necromancer's next move.}

\subsection*{\MakeUppercase{Scenario length}}

This Scenario plays out over 8 Rounds.

\subsection*{\MakeUppercase{Player setup}}

\textbf{Faction:} Castle

\textbf{Faction Hero:} The same Hero you played in Scenario 2

\textbf{Starting Resources:} The resources you ended Scenario 2 with, plus 10 \svg{gold}

\textbf{Starting Income:} 10 \svg{gold}, 0 \svg{building_materials}, 0 \svg{valuables}

\textbf{Starting Units:} The army you ended Scenario 2 with

\textbf{Town Buildings:} \svgunit{bronze} Dwelling, \svgunit{silver} Dwelling, Citadel

\textbf{Bonus:} Begin the Scenario at Hero Level 3, but do not take new Ability Cards.

\subsection*{\MakeUppercase{AI Hero setup}}

\textbf{Faction:} Necropolis

\textbf{Enemy Army:} A Pack of Skeletons, a Pack of Zombies, a Pack of Liches, a Pack of Dread Knights

\textbf{Enemy Deck:} 3 x Might Cards, 3 x Magic Cards, 1 x Skill Card

\textbf{Enemy Spells:} 2 x Lightning Bolt, 1 x Curse

\textbf{Enemy Skills:} Cloak of the Undead King IV and VI, the enemy will play whichever it can

\subsection*{\MakeUppercase{Map setup}}

Take the following Map Tiles and arrange them as shown in the Scenario map layout:

\textbf{1 × Starting (I) Map Tile}
\begin{itemize}
    \item 1 x Necropolis (S1)
\end{itemize}

\textbf{2 × Far (II--III) Map Tile}
\begin{itemize}
    \item 2 x random Tiles from Necropolis or Dungeon (F1, F2, F4, F5, F7, F8)
\end{itemize}

\textbf{4 × Near (IV--V) Map Tile}
\begin{itemize}
    \item 2 x Necropolis (N1, N4)
    \item 2 x Dungeon (N2, N5)
\end{itemize}

\textbf{1 × Center (VI--VII) Map Tile}
\begin{itemize}
  \item 1 × Dragon Utopia Tile (C1)
\end{itemize}

\subsection*{\MakeUppercase{Heroes placement}}

Place your Castle Hero on the center Field of the Necropolis start Tile S1.

Place a Necropolis Hero on the bottom left Field of the leftmost Far Tile.
Reveal the Tile when the game begins.

\subsection*{\MakeUppercase{Victory Conditions}}

\begin{itemize}
  \item Defeat the enemy army
\end{itemize}


\subsection*{\MakeUppercase{Defeat Conditions}}
\begin{itemize}
  \item You lose one combat encounter
  \item You run out of time at the end of the 8th Round
\end{itemize}
\end{multicols*}

\begin{multicols}{2}

\subsection*{\MakeUppercase{Timed Events}}

\textbf{\nth{1} Round:}
\begin{itemize}
  \item Read ``The Race'' section
\end{itemize}

\textbf{When you complete the Scenario:}
\begin{itemize}
  \item Read ``Checkmate'' section
\end{itemize}

\subsection*{\MakeUppercase{Additional rules}}

During this Castle campaign Scenario, the following rules apply:

\begin{itemize}
  \item The Glory of Erathia building is unavailable.
  \item Each time you visit a Trading Post, draw the top Cards from the \svgunit{bronze} bronze, \svgunit{silver} silver, and \svgunit{golden} golden neutral enemy Decks.
  You may recruit any, or all of them by paying their cost.
  Shuffle any un-hired ones back into their Decks.

  \item After defeating the neutral army at the Dragon Utopia, add a random dragon from the \svgunit{azure} Azure neutral Deck to your army.
  \item The enemy army follows the normal AI movement rules, stopping to capture Settlements and Mines, with its ultimate goal to reach the Dragon Utopia.
  \item If the enemy army reaches the Dragon Utopia before you do, flip the Pack side of Dread Knights to Few and add a Pack of Ghost Dragons.
\end{itemize}

\begin{tikzpicture}[overlay]
    \node at (-1.0, -8.5) {
      \includegraphics[width=0.70\paperwidth]{\_assets/maps/castle_two_knights_defense.png}
    };
  \end{tikzpicture}
\end{multicols}

\newpage

\begin{multicols*}{2}

\subsection*{\MakeUppercase{The story}}

\textbf{The Race}

Stacks of dry, cracked parchment sat next to dusty tomes on a table's blackened, rough-cut stone slab.
Catherine slumped over the numerous texts stashed throughout the necromancer's chambers while her eyes frantically scanned the pages.
She was searching for any clue to her enemy's plans.

After fruitless hours, she huffed and sent a moldering volume crashing through a stack of paper as exhaustion threatened to overcome her.
She was about to spin and leave the room when the corner of her eye caught a particular scribbling, revealed by her destructive act.
An ornate diagram, hastily scribbled, depicted what Catherine supposed might be some kind of necromantic summoning ritual.
Pulling the frail parchment closer, while most of it didn't make sense to her, it didn't take a mage to know what the dark drawings detailed.
They were the ritual instructions for raising Ghost Dragons from dead dragons.

Then it dawned on her.
Old tales told of an ancient Dragon Utopia in the badlands.
Vhidhax must be attempting to overtake the majestic creatures and use them against her in a last-ditch effort.
She frowned at the thought of her weakened enemy being able to defeat the dragons.
But if he did… If she missed something, he would have newfound strength enough to overpower her battered forces.
She clenched her fist, crushing the frail paper to dust before rushing to the doorway to gather who she could.

To her surprise, the necropolis' courtyard was abuzz with energy.
She placed her hands on the overlook's railing and peered over the edge to spy a host of knights in the color and flair of various baronies.

Rion came rushing up to her a moment later, intent on delivering news of their new guests.
``Queen Catherine!'' he said excitedly.
``Lordships around Steadwick heard of the tragedy and have sent their champions to aid us.
The knights spent days tracking and catching up to us, and they mean to join us and see the campaign through.''

Catherine smiled wide.
``Then tell them to get ready; the race is on.''

\textcolor{darkcandyapplered}{Add a Few Champions to your army.}

\textbf{Checkmate}

The broken remains of Vhidhax looked no grander than any of his many minions.
Necromantic energy had stripped the flesh from him long ago; without that power, his presence and ill-willed malevolence were gone as well.

Catherine wiped blood, sweat, and dirt from her face.
Rion had been right; the necromancer was not her father, but perhaps his slaying would ease the sting of her family discord and the sting of betrayal, at least a little.

Rion approached, and Catherine turned before embracing him in a rare moment of emotion.
The last time he'd seen his queen like this was when they were in Steadwick together, and he looked a little embarrassed as he was lost for words.
The two studied each other momentarily, silently checking on each other.
Rion thought she looked much older and weathered from the ordeal, but he was happy to see a twinkle in his queen's eyes again.

Her duty fulfilled and her seat secure, Catherine would see Steadwick rebuilt with Erathia just a little safer… for now.

\begin{itemize}
    \item \textcolor{darkcandyapplered}{If neither you or the enemy reached the Dragon Utopia, read ``A Queen's Right''}
    \item \textcolor{darkcandyapplered}{If the enemy army reached the Dragon Utopia before being defeated, read ``A Queen's Lament''}
    \item \textcolor{darkcandyapplered}{If you reached the Dragon Utopia and defeated the dragons within before defeating the enemy, read ``A Queen's Fury''}
\end{itemize}

\textbf{A Queen's Fury}

Catherine's hard eyes scanned the badlands from a window high on the Dragon Utopia.
While coming to blows with the famed dragon guardians was regrettable, she'd brought even them to heel, and the palace was hers.
No one would challenge her reign, not with the badlands under her control or the remaining dragons in her retinue.

Worn and tired faces looked up at her outside the fortress, like beggars asking for scraps.
What was left of who she'd brought into the hostile landscape was silently requesting the right to return home, yet the badlands were an enticing prize, and the bloodlust whipped up in the Queen of Erathia hadn't subsided yet.

It took much convincing from Rion before Catherine finally softened.
Reports from Steadwick were thrust in her face, and the mage's gentle nature reminded her that there were still those in her lands who needed her guidance.
So, finally, she ordered only a select few to watch over the badlands with the remaining mercenaries and dragons while she returned home with the rest.

\textbf{A Queen's Right}

A strange, deep calling swelled within Catherine.
She turned her focus to the majestic Dragon Utopia, with its rust-colored walls and pointed turrets.
It was an odd place for dragons to call home, yet its sheer presence was undeniable.

Catherine approached slowly, with the remains of her battered forces following cautiously.
Reaching the foot of the fortress, it was only a moment before the majestic snout of an azure dragon crawled into view from the large, curved, open entry to the stony fortress.
The rest of the creature soon followed, and with a stretch of wings, it stared down at the human queen.
While many fell to their knees, Catherine looked up to meet the dragon's gaze, and the two shared a moment of understanding between royalty.
It appeared the dragon knew what fate it had been spared from, and in thanks, it bowed to the human.
After a reflective moment, Catherine nodded, then turned, satisfied with the interaction.

His steed at a trot, Rion quickly approached.
``Well? Are we going inside?'' he asked curiously.

``There's no need,'' Catherine replied.
``I saw it in her eyes; the dragon knows what we did today.
We have their thanks and perhaps a favor to call on in the future.
Gods know we need it…''

Catherine drew her sword and reared her own steed high, presenting herself before her remaining forces with the dragon silhouetted behind her.
After a rousing speech that stirred the hearts of man and beast alike, she galloped to the rear lines to turn the formation toward home.

\textbf{A Queen's Lament}

Reality set in after the adrenaline and euphoria of a hard-fought victory wore off.
The battlefield was a nightmare of horrors.
Ghost Dragon corpses faded away as the last of their power died with them, and the scattered, broken bodies of both sides lay in clumps of gory remains.
The cost had been high…

Catherine regretted she couldn't save the dragons from their fate.
If things had been different, they might have been strong allies or at least sentinels to keep the local badland creatures in check.
Now, it was hard to tell what would leak from the unchecked wilds into other regions or townships under her charge.
Caution had won her the day, but she wondered if it was enough.
Would actions like this cause her downfall one day?

Home felt distant in the haze of fury that Catherine had been prisoner to.
As she trotted past the remains of those who had fought for her, her heart swelled with emotion.
It was war, and loss was inevitable.
She turned to the remaining few and threw a weak hand forward out of the badlands.
It was time to pick up the pieces.

\end{multicols*}



\addscenariogroup{\campaignstitle}{\layout/campaigns.png}

\clearpage

\input{\campaignspath/inferno_devilish_plan.tex}

\clearpage

% !TeX spellcheck = en_US
\addscenariosection[subsection]{1}{Inferno Campaign $-$ Dungeons and Devils}{2. Steadwick's Fall}{\images/firewall.png}

\begin{multicols*}{2}

\textbf{Author:} Tm335

\textbf{Source:} \href{https://discord.com/channels/740870068178649108/1246353361456861276/1246353361456861276}{Archon Studios Discord}

\textit{Catherine Ironfist has enlisted aid from Bracada and AvLee.
She knows we are close to Steadwick.
We must occupy Steadwick before she arrives.
Once we own Erathia's capitol, not even Catherine Ironfist will wrench it from our hands.}

\subsection*{\MakeUppercase{Scenario Length}}

This Scenario plays out over 16 Rounds.

\subsection*{\MakeUppercase{Player Setup}}

\textbf{Faction:} Inferno

\textbf{Faction Hero:} Choose any

\textbf{Starting Resources:} 15 \svg{gold}, 1 \svg{building_materials}, 1 \svg{valuables}

\textbf{Starting Income:} 10 \svg{gold}, 2 \svg{building_materials}, 0 \svg{valuables}

\textbf{Starting Units:}

\begin{itemize}
  \item A Few Troglodytes
  \item A Few Evil Eyes
  \item A Pack of Familiars
\end{itemize}

\textbf{Town Buildings:} \svgunit{bronze} Dwelling, \svgunit{silver} Dwelling

\textbf{Bonus:} Choose one of the following options:
\begin{itemize}
  \item Add a Pack of Harpies to your hand.
  \item Add a Pack of Magogs to your hand.
  \item Add an Ammo Cart to your hand.
  \item Search (4) the Spell Deck.
\end{itemize}

\subsection*{\MakeUppercase{AI Hero Setup}}

\textbf{Faction:} Castle

\textbf{Enemies:} General Kendal, Charging Heroes

\textbf{General Kendal's Army:} A Pack of Archangels, a Pack of Champions, a Pack of Zealots, a Pack of Crusaders, a Pack of Griffins, a Balista War Machine

\textbf{General Kendal's Deck:} 5 × Might Card, 1 × Magic Card, 3 × Skill Card

\textbf{General Kendal's Spell Deck:} 2 × Haste Spell Card

\textbf{General Kendal's Skill:} Artillery Ability Card\footnote{For General Kendal, the Artilery Ability Card always resolves the Expert effect and the Ballista War Machine Activates every time the Artillery Ability Card is drawn, as well as at the beginning of a Combat Round.}

\textbf{Charging Heroes' Factions:} Rampart, Tower, Castle

\textbf{Charging Heroes' Armies:} Neutral Army is one Level higher than your Hero Level (Max Level VI)\footnote{See page 35, ``Field Difficulty Level Table'' in the Core Rulebook, for further details on the number of Neutral Units you have to draw for this Neutral Army.}.

\textbf{Charging Heroes' Deck:} 2 × Might Card, 2 × Magic Card

\textbf{Charging Heroes' Spell Deck:} 2 × Slow Spell Card\footnote{All the Charging Heroes' Enemies use the same AI and Spell Decks. Reset them after every Combat.}.

\subsection*{\MakeUppercase{Map Setup}}

Take the following Map Tiles and set them up as shown in the Scenario map layout:

\textbf{2 × Starting (I) Map Tile}
\begin{itemize}
  \item 1 × Inferno (S6)
  \item 1 × Castle (S3)
\end{itemize}

\textbf{3 × Far (II--III) Map Tile}
\begin{itemize}
  \item 1 × Castle (F3)
  \item 1 × Rampart (F10)
  \item 1 × Dungeon (F2)
  \item 1 × Tower (\#F1)
  \item 1 × Rampart (choose from: F11, F12)
  \item 1 × Necropolis (choose from: F4, \#F6, F7)
\end{itemize}

\textbf{2 × Near (IV--V) Map Tile}
\begin{itemize}
  \item 2 × Castle (N3, choose from: \#N3, N6)
  \item 1 × Rampart (N8)
  \item 1 × Necropolis (N4)
\end{itemize}

\subsection*{\MakeUppercase{Heroes Placement}}

The Enemy Hero General Kendal is represented by one Castle Faction Hero model and appears on the center Field of the S3 Starting Map Tile.

The Castle Charging Hero is represented by one Castle Faction Hero model and appears on (and owns) the Settlement of the F3 Map Tile.

The Rampart Charging Hero is represented by one Rampart Faction Hero model and appears on (and owns) the Settlement of the F10 Map Tile.

The Tower Charging Hero is represented by one Tower Faction Hero model and appears on (and owns) the Settlement of the \#F1 Map Tile.

Place your Main Hero on the center Field of the Inferno Starting S6 Map Tile.

Place your Secondary Hero, represented by one Dungeon Faction Hero model of your choice, on the Settlement of the F2 Map Tile. This Settlement produces no resources.

\subsection*{\MakeUppercase{Victory Conditions}}

Defeat all 3 Enemy Controlled Settlements (F3, F10, \#F1), and capture Steadwick (S3) before
Queen Catherine Ironfist arrives at the end of the Round 16.

\subsection*{\MakeUppercase{Defeat Conditions}}

You lose one Combat encounter with your Main Hero (Surrendering costs 10 \svg{gold}, and does not count as a defeat).

You lose your Faction Town on the S6 Map Tile.

You run out of time -- you have time till the end of the Round 16.

\subsection*{\MakeUppercase{Timed Events}}

\textbf{\nth{1} Round:}
\begin{itemize}
  \item Read: ``You are to be congratulated on your progress so far.
    You have laid waste to Eastern Erathia, and are now within striking distance of the Erathian
    capital of Steadwick. You must capture the capital quickly!''
\end{itemize}

\textbf{\nth{2} Round:}
\begin{itemize}
  \item Read: ``Not only have Bracada and AvLee sent reinforcements, but we have received news that
    Queen Catherine Ironfist is marching a sizeable army from the south. We must control the capital and its
    garrisons before she arrives.

    You have just received a report on the progress of Queen Catherine.
    Forces from Nighon and Eeofol are attempting to delay her march to Steadwick,
    but doubt that they can delay her more than two or three months.''
\end{itemize}

\textbf{\nth{5} Round:}
\begin{itemize}
  \item Read: ``You receive a report from the south. Queen Catherine's forces have been sufficiently delayed,
    allowing you at least two months more to reach the capitol, but our own forces have suffered significant
    losses. Do not let their sacrifice go to waste.''
\end{itemize}

\textbf{\nth{10} Round:}
\begin{itemize}
  \item For any Enemy Settlements (F3, F10, \#F1) that have not been defeated, an additional Enemy
    Charging Hero of that Faction appears on each Settlement at the beginning of this Round.
\end{itemize}

\textbf{\nth{11} Round:}
\begin{itemize}
  \item Read: ``You receive a report from the south. Our forces continue to throw themselves in the path of
    Queen Catherine's armies, yet she continues to march northward. You have, at most, three or four weeks
    before she can reach the capital.''
\end{itemize}

\textbf{\nth{13} Round:}
\begin{itemize}
  \item Warning: ``Queen Catherine's march continues -- her forces are just two weeks away. If you do not hurry,
    we will not have time to secure the capital before her arrival.''
\end{itemize}

\textbf{\nth{15} Round:}
\begin{itemize}
  \item Warning: ``Your scouts report sighting Queen Catherine's army seven days to the southwest. If she
    reaches the capitol before you, all is lost.''
\end{itemize}

\textbf{\nth{16} Round:}
\begin{itemize}
  \item If Steadwick has not been taken, read: ``This morning, a massive army lead by Queen Catherine
    Ironfist arrived at the Erathian capitol of Steadwick. We have no choice but to retreat our forces.
    You have failed us... miserably.''
\end{itemize}

\textbf{When you complete the Scenario:}
\begin{itemize}
  \item Read: ``Congratulations! You captured Steadwick and are victorious!''
\end{itemize}

\subsection*{\MakeUppercase{Additional Rules}}

During this ``Inferno'' Campaign Scenario, the following rules apply:

\begin{itemize}
    \item The General Kendal Hero does not move and only waits in their Town. They start with Walls, a Gate,
      and an Arrow Tower on their side of the Combat Board.
    \item The difficulty Level of every Combat encounter on the map increases by one till the end of the Scenario
      (see page 35, ``Field Difficulty Level Table'' in the Core Rulebook).
    \item The Enemy Charging Heroes have only 2 Movement Points, instead of 3. They ignore everything else
      (including Mines and Settlements if possible) and go straight for the player's Faction Town (on Map Tile S6).
      They do not pursue the player directly, but if they happen to be on the same Map Tile, they will
      attack the player's Main Hero (not Secondary).
    \item The Enemy Charging Heroes move after the player's Turn ends.
    \item The borders on the Castle Starting Map Tile S3 cannot be crossed by any means until
      all Enemy Settlements (F3, F10, \#F1) have been defeated.
    \item Defeating Enemy Charging Heroes provides 2 \svg{valuables}.
    \item Obelisks give you 1 \svg{valuables} and function as a ``Castle
      Gate'' location (only if you have the ``Castle Gate'' built in your Town).
\end{itemize}

\vspace*{\fill}
\begin{center}
  \framedimage[0.9\linewidth]{\art/demon.jpg}
\end{center}
\vspace*{\fill}

\end{multicols*}

\begin{tikzpicture}[remember picture, overlay]
  \node(bg)[anchor=center, yshift=37em, opacity=0.17] at (current page.south) {
    \includegraphics[width=0.85\paperwidth, keepaspectratio]{\art/fire_shield.png}
  };
  \node(map)[anchor=center] at (current page.center) {
    \includegraphics[width=\textwidth]{\maps/inferno_steadwicks_fall.png}
  };

  \node at (3, -11.5) {\large{{\textbf{\textcolor{darkcandyapplered}{N3}}}}};
  \node at (4.4, -18.6) {\large{{\textbf{\textcolor{darkcandyapplered}{\#F1}}}}};
  \node at (15, -5.2) {\large{{\textbf{\textcolor{darkcandyapplered}{F10}}}}};
  \node at (17.3, -11.5) {\large{{\textbf{\textcolor{darkcandyapplered}{S6}}}}};
  \node at (8.8, -18.8) {\large{{\textbf{\textcolor{darkcandyapplered}{F3}}}}};
  \node at (15.9, -18.8) {\large{{\textbf{\textcolor{darkcandyapplered}{F2}}}}};
  \node at (2, -4.5) {\large{{\textbf{\textcolor{darkcandyapplered}{Steadwick}}}}};
  \node at (2, -5.2) {\large{{\textbf{\textcolor{darkcandyapplered}{S3}}}}};
  \node at (7.5, -5.4) {\large{{\textbf{\textcolor{darkcandyapplered}{Rampart}}}}};
  \node at (7.5, -6.1) {\large{{\textbf{\textcolor{darkcandyapplered}{Map Tiles}}}}};
  \node at (12.4, -17.8) {\large{{\textbf{\textcolor{darkcandyapplered}{Necropolis}}}}};
  \node at (12.4, -18.5) {\large{{\textbf{\textcolor{darkcandyapplered}{Map Tiles}}}}};
  \node at (2, -12.3) {\large{{\textbf{\textcolor{darkcandyapplered}{Castle}}}}};
  \node at (2, -13) {\large{{\textbf{\textcolor{darkcandyapplered}{Map Tiles}}}}};
\end{tikzpicture}


\clearpage

\input{\campaignspath/inferno_deal_with_the_devil.tex}

\clearpage

% !TeX spellcheck = en_US
\cleardoublepage\phantomsection\addcontentsline{toc}{section}{\protect\numberline{} {} {} {} {}The Queen's Gambit}
\addscenariosection[subsection]{1}{Castle Campaign $-$ The Queen's Gambit}{1. Greek Gift}{\images/catherine.png}

\begin{multicols*}{2}

\textbf{Author:} Tyson Heckert

\textbf{Source:} \href{https://travelledtales.com}{travelledtales.com}

\textit{Steadwick is free, but that doesn't mean it is safe.
Dark powers still roam Erathia, and one now moves its pieces into position.
While Catherine grapples with her responsibility — and her demons — she must be a bulwark against evil.}

\subsection*{\MakeUppercase{Scenario length}}

This Scenario plays out over 8 Rounds.

\subsection*{\MakeUppercase{Player setup}}

\textbf{Faction:} Castle

\textbf{Faction Hero:} Any Castle Hero. If you are not Catherine, you are in her entourage for this story.

\textbf{Starting Resources:} 10 \svg{gold}, 3 \svg{building_materials}, 1 \svg{valuables}

\textbf{Starting Income:} 10 \svg{gold}, 0 \svg{building_materials}, 0 \svg{valuables}

\textbf{Starting Units:}
\begin{itemize}
  \item A Few Halberdiers, a Few Marksmen
\end{itemize}

\textbf{Town Buildings:} \svgunit{bronze} Dwelling, Citadel

\subsection*{\MakeUppercase{AI Hero setup}}

\textbf{Faction:} Necropolis

\textbf{Enemy Army:} A Pack of Skeletons, a Pack of Zombies, a Few Dread Knights, a Few Ghost Dragons

\textbf{Enemy Deck:} 4 x Might Cards

\textbf{Special:} Place the Cloak of the Undead King IV and VI Cards aside

\subsection*{\MakeUppercase{Map setup}}

Take the following Map Tiles and arrange them as shown in the Scenario map layout:

\textbf{1 × Starting (I) Map Tile}
\begin{itemize}
    \item 1 x Castle (S3)
\end{itemize}

\textbf{3 × Far (II--III) Map Tile}
\begin{itemize}
    \item 3 x Castle (F3, F6, F9)
\end{itemize}

\textbf{2 × Near (IV--V) Map Tile}
\begin{itemize}
    \item 2 x Castle (N3, N6)
\end{itemize}

\subsection*{\MakeUppercase{Heroes placement}}

Place your Castle Hero on the center Field of the S3 Castle starting map Tile.

Place a Necropolis Hero on the center Field of the N3 Castle Tile.

\subsection*{\MakeUppercase{Victory Conditions}}

Defeat the enemy army.

\subsection*{\MakeUppercase{Defeat Conditions}}
\begin{itemize}
  \item You lose one combat encounter
  \item You run out of time at the end of the 8th Round
\end{itemize}
\end{multicols*}

\newpage

\begin{multicols}{2}

\subsection*{\MakeUppercase{Timed Events}}

\textbf{\nth{1} Round:}
\begin{itemize}
  \item Read ``Distant Threat'' section
\end{itemize}

\textbf{If entering the settlement on F3:}
\begin{itemize}
  \item Read ``Reverent Allies'' section
\end{itemize}

\textbf{Before combat with the enemy Hero:}
\begin{itemize}
  \item Read ``Bait and Switch'' section
\end{itemize}

\textbf{When you complete the Scenario:}
\begin{itemize}
  \item Read ``The Chase Begins'' section
\end{itemize}

\subsection*{\MakeUppercase{Additional rules}}

During this Castle campaign Scenario, the following rules apply:

\begin{itemize}
  \item The enemy army does not move
  \item Your Hero's maximum Level is 4
  \item The Glory of Erathia building is unavailable
  \item The Settlement on F3 does not have neutral enemies
\end{itemize}

\columnbreak

\includegraphics[width=\linewidth, keepaspectratio]{\art/griffin.png}

\end{multicols}

\begin{tikzpicture}[overlay]
  \node at (8.6, -5.3) {
    \includegraphics[width=\textwidth]{\_assets/maps/castle_greek_gift.png}
  };
  \node at (10, -2.0) {\large{{\textbf{\textcolor{darkcandyapplered}{F3}}}}};
\end{tikzpicture}

\newpage

\begin{multicols*}{2}

\subsection*{\MakeUppercase{The story}}

\textbf{Distant Threat}

Catherine Ironfist rubbed her tired, worn thumb across the image of her father, King Gryphonheart, in a locket.
She wanted to remember the good, valiant parts of his life rather than his actions in unlife.
Her gaze turned to her cracked skin and mangled thumbnail, and it reminded her of how hard she had fought to be where she was.

The weight of Erathia rested on her shoulders.
It had only been weeks since the siege and recapture of Steadwick, and resources, as well as manpower, were in short supply.
Still, with time, the stony seat of the castle would be sufficient to further liberate the realm.

One such problem was weighing on her mind.
After replacing the locket in her hand with a missive, she re-read a particular entry about a rising necromancer named Vhidhax.
Rumors were that the fiend had stolen precious artifacts that further enhanced his power over the dead.
While rumors are usually best left to fishwives and children, this one aligned with information about an army growing in strength, and she dreaded the thought of the two problems colliding and crashing over her weary crown.

Late that evening, Catherine was venting to Rion about the relative drudgery of rebuilding Steadwick.
She'd just about fallen asleep in her chair when a heavy knock turned her attention to the room's large door that sported intricately carved shields from various long-gone baronies.

The wooden portal flung open before she could spring up, revealing a wide-eyed and panicked guard.
No words came from the man; whatever haunted him seemed absolute.
Instead, he merely pointed with a trembling hand toward the window across the room.

Catherine shot a look at Rion before the two darted toward the open air of the rectangular window.
What they saw shocked them, too, and while Catherine's heart sank, she gripped Rion's shoulder with surprising strength.
When he looked back at her with a slight wince, she only returned a stony look of resolve.

Danger had come to Steadwick again.
Atop a hill in the far distance, overlooking many of the defenseless farmsteads Catherine had in her charge, a predator swept its gaze like a wolf might hungrily eye sheep.
A magnificently terrifying azure hue wafted from a Ghost Dragon's massive, extended wings like an ethereal heat haze.

\textbf{Reverent Allies}

While Catherine trusted her horse, having fought tooth and nail with the battle-hardened steed for as long as she could remember, it shuddered in hesitation as they approached the majestic, columned main building of the settlement.
Catherine had also felt the almost unsettling holy power in the area when, suddenly, the sound of rushing wings caught her attention.
Her gaze snapped upward to catch sight of something approaching.

The early morning sun was bright, and the figure that swooped low was in front of it.
The silhouette of wings was visible, and Catherine thought it to be a Gryphon at first.
However, as it closed in on them, it became apparent it was the shape of a man with wings instead.
It caught Catherine's horse off guard, and she had to catch the reins and wrench them hard to steady herself and calm the frightened animal before it could fling her from her seat.

With lightning speed, the figure landed mere feet from them to reveal the stubbled, handsome face of a young man with golden, curled locs and impressively robust bronze armor sculpted to his muscular form.
He was one of the famed Archangels to have stayed near Steadwick after Catherine's victory.
While the forces behind her fell to their knees in reverence, she only nodded with stern resolve.

With a deeper, and more reverberating voice than any man, the angel was first to speak.
``Queen Catherine.
I expected you might come when I spied the rot taking form in the hills.''

Catherine responded with a smile.
It was good to see she still had allies; her forces would need bolstering to take on the undead.
``It's good to see you, my friend,'' she said.
``It seems Steadwick is threatened again, and I ride to root out the evil before it can seed.
Would you join me at my side?''

``Of course, my queen!'' the angel barked.
``Combined, no evil can thwart our advance!'' As the final words left his lips and began to echo, he drew his weapon.
The longsword ignited as it left its sheath, and its blazing glory had most of those kneeling averting their eyes.
Immediately after, he took to the air and began circling the area again.
Catherine smiled wide as she watched more angels appear, then looked back to her forces and extended an open palmed arm forward in a command to march.

\textcolor{darkcandyapplered}{Add a Few Archangels to your army.}

\textbf{Bait And Switch}

Catherine couldn't wait to drive her sword through the heart of the Ghost Dragon, then lop the head from the one commanding it.
She pushed her powerful mount forward, charging toward the waiting, taunting foes.
While the great number of shambling undead, mostly animated citizens from nearby townships, was impossible to ignore, they wouldn't stop her.
With a thunderous crash, she slammed past the forward lines, knocking off limbs and cracking bones.

Seconds later, she was halfway through the mass of bodies when the stench of rotten flesh and the true immensity of her enemy's numbers hit her.
Dread Knights flanked the dragon she'd been racing towards.
Worse still, a robed, decrepit-looking figure was watching from the rear of the enemy lines.
Despite its wretched appearance, it had an immensely powerful presence to it.
A stinging tingle shot up Catherine's spine as she felt, more than saw, its eyes return her gaze.
More than she could take on alone, she swung her steed in a circle to reform with her mounted allies, then looked to the sky just as an Archangel came crashing down onto the dragon.

Shockingly, instead of slamming into the beast, the angel flew through it like it was made of air.
He hit the ground with a crunching thud before, suddenly, the dragon's image began to shift and shimmer, then disintegrate.
The Dread Knights did the same, and replacing them were throngs of further undead that now threatened to encircle Catherine and her forward forces.

Catherine's eyes shot toward what she believed was the necromancer.
It stood motionless a moment longer before cackling laughter cut through the air, and its image wafted away into the night air just as the others had.

It was a trap.
The necromancer and his strongest forces were not there, and Catherine would have to fight her way out from under a crush of undead, wasting precious time and lives.

\begin{itemize}
  \item \textcolor{darkcandyapplered}{Replace the Ghost Dragons and Dread Knights with \svgunit{bronze} neutral Zombies and Skeletons.}
  \item \textcolor{darkcandyapplered}{Play ``Cloak of the Undead King IV'' and ``Cloak of the Undead King VI'' onto the packs of Zombies and Skeletons, respectively.}
  \item \textcolor{darkcandyapplered}{Remove the AI Might and Magic Deck from the game, the undead fight on their own.}
\end{itemize}


\textbf{The Chase Begins}

Catherine huffed, sweat beading on her forehead as the last undead fell with wet thuds.
Her forces were strong, and simple reanimations weren't the kind of threat to stop them.
She just hoped… Her stomach churned as her gaze turned back to Steadwick.

An orange glow, far too bright for torchlight, was coming from the castle.
As she watched in horror, thick black smoke began to rise from the towers and parapets.
Steadwick was burning.

Rion quickly caught up to her, his horse panting as hard as he was.
He began to speak but couldn't find the words as his mind raced.
Instead, he placed a hand on his queen's shoulder.

The necromancer had played his opening move and won.

Just before Catherine turned to her regrouping forces, she spotted the spectral blue of the real Ghost Dragon in the distance.
Its work destroying Steadwick's defenses had been finished.

Archangels circled overhead, and Catherine threw her head to the sky, tears welling at the thought of losing what she had worked so hard to build.
``Find him!'' she roared, addressing whatever gods were listening as well as the angels above.

``Is that… wise?'' Rion asked softly.

Catherine responded with fire in her eyes.
``What good is rebuilding a ruin when the fiend will just continue to play his tricks?'' she said.
``No.
If we want to save Steadwick, we have to end him before he can further strengthen himself.''

\begin{itemize}
  \item \textcolor{darkcandyapplered}{Remove Archangels from your army.}
  \item \textcolor{darkcandyapplered}{Keep the rest of your forces and resources for the next Scenario.}
  \item \textcolor{darkcandyapplered}{Reset your Hero per the normal campaign rules.}
\end{itemize}

\columnbreak

\begin{center}
  \framedimage[0.8\linewidth]{\art/ghost_dragon.jpg}
\end{center}

\end{multicols*}


\clearpage

% !TeX spellcheck = en_US
\addscenariosection[subsection]{1}{Castle Campaign $-$ The Queen's Gambit}{2. Sicilian Dragon}{\images/catherine.png}

\begin{multicols*}{2}

\textbf{Author:} Tyson Heckert

\textbf{Source:} \href{https://travelledtales.com}{travelledtales.com}

\textit{Convinced that the only way to save Steadwick is to attack Vhidhax, Catherine powers her small force ahead.
Following the Archangels to a distant, inhospitable land, she must now navigate her way through.}

\subsection*{\MakeUppercase{Scenario length}}

This Scenario plays out over 11 Rounds.

\subsection*{\MakeUppercase{Player setup}}

\textbf{Faction:} Castle

\textbf{Faction Hero:} The same Hero as Scenario 1

\textbf{Starting Resources:} The resources you ended with in Scenario 1

\textbf{Starting Income:} 10 \svg{gold}, 0 \svg{building_materials}, 0 \svg{valuables}

\textbf{Starting Units:} The army you ended with in Scenario 1

\textbf{Town Buildings:} None

\textbf{Bonus:} Begin the Scenario at Hero Level 2, but do not take new Ability Cards.

\subsection*{\MakeUppercase{AI Hero setup}}

\textbf{Faction:} Necropolis

\textbf{Enemy Army:} A Pack of Skeletons, a Pack of Zombies, a Pack of Liches, a Few Dread Knights, a Few Ghost Dragons

\textbf{Enemy Deck:} 3 x Might Cards, 3 x Magic Cards, 1 x Skill Card

\textbf{Enemy Spells:} 2 x Lightning Bolt, 1 x Curse

\textbf{Enemy Skills:} Cloak of the Undead King IV and VI, the enemy will play whichever it can

\textbf{Neutral Enemy Deck:} 4 x Might Cards

\subsection*{\MakeUppercase{Map setup}}

Take the following Map Tiles and arrange them as shown in the Scenario map layout:

\textbf{1 × Starting (I) Map Tile}
\begin{itemize}
    \item 1 x Necropolis (S1)
\end{itemize}

\textbf{3 × Far (II--III) Map Tile}
\begin{itemize}
    \item 3 x random Tiles from Necropolis or Dungeon (F1, F2, F4, F5, F7, F8)
\end{itemize}

\textbf{4 × Near (IV--V) Map Tile}
\begin{itemize}
    \item 2 x Necropolis (N1, N4)
    \item 2 x Dungeon (N2, N5)
\end{itemize}

\textbf{1 × Center (VI--VII) Map Tile}
\begin{itemize}
  \item 1 × Grail Tile (C2)
\end{itemize}

\textbf{\MakeUppercase{Note:}} Place Tile S1 aside, face up.

\subsection*{\MakeUppercase{Heroes placement}}

Place your Castle Hero on the empty Field of the rightmost Far Tile.

Place a Necropolis Hero on the center Field of the S1 Necropolis Tile.

\subsection*{\MakeUppercase{Victory Conditions}}

\begin{itemize}
  \item Find the Archangels
  \item Defeat the enemy army
\end{itemize}


\subsection*{\MakeUppercase{Defeat Conditions}}
\begin{itemize}
  \item You lose one combat encounter
  \item You run out of time at the end of the 11th Round
\end{itemize}

\end{multicols*}

\begin{multicols}{2}

\subsection*{\MakeUppercase{Timed Events}}

\textbf{\nth{1} Round:}
\begin{itemize}
  \item Read ``Into the Badlands'' section
\end{itemize}

\textbf{Visiting the first Obelisk:}
\begin{itemize}
  \item Read ``The First Relic'' section
\end{itemize}

\textbf{Visiting the second Obelisk:}
\begin{itemize}
  \item Read ``Showdown'' section
\end{itemize}

\textbf{When you complete the Scenario:}
\begin{itemize}
  \item Read ``The Escape'' section
\end{itemize}

\subsection*{\MakeUppercase{Additional rules}}

During this Castle campaign Scenario, the following rules apply:

\begin{itemize}
  \item The enemy army does not move
  \item Your Hero's maximum Level is 6
  \item All buildings are unavailable, building is not possible
\end{itemize}

\columnbreak

\begin{itemize}
\item All neutral enemies fight using the Neutral Enemy Deck.
\item Each time you visit a Trading Post, draw the top Cards from the \svgunit{bronze} bronze, \svgunit{silver} silver, and \svgunit{golden} golden neutral enemy Decks.
You may recruit any, or all of them by paying their cost.
Shuffle any un-hired ones back into their Decks.

\item The Grail may be sold at a trading post for 30 \svgunit{gold} \textit{or} \textbf{Search (3)} the relic Deck, twice.

\item When fighting the enemy at thier Town, they have a Citadel.
Don't forget to add the walls and arrow tower to make the battle a siege.
\end{itemize}

\end{multicols}

\begin{tikzpicture}[overlay]
  \node at (9.0, -6.0) {
    \includegraphics[width=0.80\paperwidth]{\_assets/maps/castle_sicilian_dragon.png}
  };
\end{tikzpicture}

\newpage

\begin{multicols*}{2}

\subsection*{\MakeUppercase{The story}}

\textbf{Into the Badlands}

Catherine crossed arms and leaned on her saddle horn as she peered out over the wasteland they'd been led to.
The Archangels had gone ahead of the small force she'd scraped together, and she wondered if it had been a mistake to let them go.
With the enemy stronghold still nowhere in sight, locating it would take time, and the inhospitable landscape would not be kind.

Rion pulled up next to her as he finished gulping the last contents of a water pouch.
He wiped the dribble and sweat from his face before saying anything.
``Not the best place to do any fighting.
Besides the poor conditions, I've heard the locals are known for their particular… ferocity.''

``Well then, we'll just have to bring them to heel,'' Catherine replied coldly.
The truth was she was more worried about that possibility than she let on, but she had to maintain her composure to protect the image of her strength.

``You think they'll fight for us?'' Rion asked with a raised eyebrow.

Catherine tightly gripped a heavy sack of coin by her side and heaved it into her lap so the contents jingled like tiny bells to emphasize her point.
``I think generous arrangements can be made.
For the rest, the sword will do.''

\textbf{The First Relic}

Ahead, a clearing of dry, cracked ground gave way to a large Obelisk protruding from the ground like a massive marker.
Catherine had seen enough magical artifacts in her time that she could tell the structure was a place of power.
Throwing an arm forward, she signaled Rion to proceed with an inspection.

The mage approached slowly, removing a glove to touch the cold stone of the smoothly cut sides with his skin.
Intricate runes were carved in it, detailing all manner of scripture in long-lost tongues.
As awe-inspiring as it was, equally apparent was the time-worn damage.

After a moment, he turned to Catherine.
``It's a transport beacon,'' he said.
``Like a monolith, but with a shorter range.
At least… it was.
It's damaged, and where the magic in the runes should be strong, even glowing, they're dark, and any power is long decayed.''

Catherine dismounted and slowly inspected the area before saying anything.
``Some of the damage is recent.
I can still feel residual heat; the Archangels were here.''

``How can you be sure? Surely, some would have found us by now and notified us of their progress.''

``They've never let me down,'' Catherine grunted.
``Come, let's see if other such beacons exist.''

\textbf{Showdown}

Sounds of battle perked the ears of Catherine's forces, and she urged her steed forward to catch sight of the combatants.

Rounding crags and jagged rocks jutting at all angles, another Obelisk revealed itself with the final moments of a battle taking place around it.
Several Archangels were on the ground, locked in combat with the remains of a larger undead force.

Approaching at speed, Catherine raised a fist in the air to salute the holy warriors, and several rasped their bronze chestplates in return.

``Queen Catherine, you've saved us the trouble of finding you,'' the lead angel said in his deep voice.
He gestured to the Obelisk with an open palm as the sounds of battle finally died.
``The foes meant to keep us from these structures by destroying them.
We barely managed to intercept them in time to preserve this one.''

Rion trotted past them as they spoke, intent on inspecting the intact Obelisk.
As before, he began to place his hand on the cold stone, but recoiled before making contact.
``It's active!'' he said excitedly.
``This may be the break we need.
With luck, this will take us directly to the enemy stronghold.''

Catherine smiled, reaching to clasp the arm of the angel in thanks for his work.
She looked back at the ragtag group of forces she'd gathered and hoped it would be enough for a siege.

\begin{itemize}
  \item \textcolor{darkcandyapplered}{Add a Few Archangels to your army.}
  \item \textcolor{darkcandyapplered}{You may now teleport to the enemy town on Tile S1.
  If you choose not to, you may in the future by spending one movement at this Obelisk.}
\end{itemize}


\textbf{The Escape}

With the thick, macabre, almost demonic-looking walls breached and the bulk of the necromancer's forces slain, the castle belonged to Catherine.
It was an eye for an eye, and she relished driving the final blow into the last enemy ranks still in the courtyard herself.

Unfortunately, after a frantic search of the grounds, Vhidhax himself was nowhere in sight.
Catherine raced up the walls just in time to peer over the side and catch sight of the necromancer fleeing with a sizable retinue that had quietly slipped away during the battle.
He was traveling north, further into the badlands and away from her home.
Catherine's campaign wasn't finished yet.

Catherine turned to observe the battlefield from above.
The butcher's bill had been large.
Both fresh and long-decaying bodies lay strewn about, several Archangels among them.
Rion caught up with her a minute later, and the two exchanged looks of exhaustion.

``It's done,'' Rion said through panting breaths.

``No.
The necromancer lives, and he flees north.''

``But… our forces.
The Archangels are too few to continue; what hope do we have?''

``There is always hope, my friend.
We'll use this castle as a base to resupply, and then we need to give chase before the fiend regains his strength.
Steadwick will never be safe otherwise!''

``This is foolish,'' Rion sighed.
``We don't need to fight this hard; Vhidhax is not your father, Catherine…''

\begin{itemize}
  \item \textcolor{darkcandyapplered}{Remove Archangels from your army, if possible.}
  \item \textcolor{darkcandyapplered}{Keep the rest of your army and resources for the next Scenario.}
  \item \textcolor{darkcandyapplered}{Reset your Hero per the normal campaign rules.}
\end{itemize}

\vspace{2em}
\begin{center}
  \framedimage[0.75\linewidth]{\art/necropolis_town.jpg}
\end{center}

\end{multicols*}


\clearpage

% !TeX spellcheck = en_US
\addscenariosection[subsection]{1}{Castle Campaign $-$ The Queen's Gambit}{3. Two Knights Defense}{\images/catherine.png}

\begin{multicols*}{2}

\textbf{Author:} Tyson Heckert

\textbf{Source:} \href{https://travelledtales.com}{travelledtales.com}

\textit{With Vhidhax's stronghold in her grasp, Catherine needs to discover and halt the necromancer's next move.}

\subsection*{\MakeUppercase{Scenario length}}

This Scenario plays out over 8 Rounds.

\subsection*{\MakeUppercase{Player setup}}

\textbf{Faction:} Castle

\textbf{Faction Hero:} The same Hero you played in Scenario 2

\textbf{Starting Resources:} The resources you ended Scenario 2 with, plus 10 \svg{gold}

\textbf{Starting Income:} 10 \svg{gold}, 0 \svg{building_materials}, 0 \svg{valuables}

\textbf{Starting Units:} The army you ended Scenario 2 with

\textbf{Town Buildings:} \svgunit{bronze} Dwelling, \svgunit{silver} Dwelling, Citadel

\textbf{Bonus:} Begin the Scenario at Hero Level 3, but do not take new Ability Cards.

\subsection*{\MakeUppercase{AI Hero setup}}

\textbf{Faction:} Necropolis

\textbf{Enemy Army:} A Pack of Skeletons, a Pack of Zombies, a Pack of Liches, a Pack of Dread Knights

\textbf{Enemy Deck:} 3 x Might Cards, 3 x Magic Cards, 1 x Skill Card

\textbf{Enemy Spells:} 2 x Lightning Bolt, 1 x Curse

\textbf{Enemy Skills:} Cloak of the Undead King IV and VI, the enemy will play whichever it can

\subsection*{\MakeUppercase{Map setup}}

Take the following Map Tiles and arrange them as shown in the Scenario map layout:

\textbf{1 × Starting (I) Map Tile}
\begin{itemize}
    \item 1 x Necropolis (S1)
\end{itemize}

\textbf{2 × Far (II--III) Map Tile}
\begin{itemize}
    \item 2 x random Tiles from Necropolis or Dungeon (F1, F2, F4, F5, F7, F8)
\end{itemize}

\textbf{4 × Near (IV--V) Map Tile}
\begin{itemize}
    \item 2 x Necropolis (N1, N4)
    \item 2 x Dungeon (N2, N5)
\end{itemize}

\textbf{1 × Center (VI--VII) Map Tile}
\begin{itemize}
  \item 1 × Dragon Utopia Tile (C1)
\end{itemize}

\subsection*{\MakeUppercase{Heroes placement}}

Place your Castle Hero on the center Field of the Necropolis start Tile S1.

Place a Necropolis Hero on the bottom left Field of the leftmost Far Tile.
Reveal the Tile when the game begins.

\subsection*{\MakeUppercase{Victory Conditions}}

\begin{itemize}
  \item Defeat the enemy army
\end{itemize}


\subsection*{\MakeUppercase{Defeat Conditions}}
\begin{itemize}
  \item You lose one combat encounter
  \item You run out of time at the end of the 8th Round
\end{itemize}
\end{multicols*}

\begin{multicols}{2}

\subsection*{\MakeUppercase{Timed Events}}

\textbf{\nth{1} Round:}
\begin{itemize}
  \item Read ``The Race'' section
\end{itemize}

\textbf{When you complete the Scenario:}
\begin{itemize}
  \item Read ``Checkmate'' section
\end{itemize}

\subsection*{\MakeUppercase{Additional rules}}

During this Castle campaign Scenario, the following rules apply:

\begin{itemize}
  \item The Glory of Erathia building is unavailable.
  \item Each time you visit a Trading Post, draw the top Cards from the \svgunit{bronze} bronze, \svgunit{silver} silver, and \svgunit{golden} golden neutral enemy Decks.
  You may recruit any, or all of them by paying their cost.
  Shuffle any un-hired ones back into their Decks.

  \item After defeating the neutral army at the Dragon Utopia, add a random dragon from the \svgunit{azure} Azure neutral Deck to your army.
  \item The enemy army follows the normal AI movement rules, stopping to capture Settlements and Mines, with its ultimate goal to reach the Dragon Utopia.
  \item If the enemy army reaches the Dragon Utopia before you do, flip the Pack side of Dread Knights to Few and add a Pack of Ghost Dragons.
\end{itemize}

\begin{tikzpicture}[overlay]
    \node at (-1.0, -8.5) {
      \includegraphics[width=0.70\paperwidth]{\_assets/maps/castle_two_knights_defense.png}
    };
  \end{tikzpicture}
\end{multicols}

\newpage

\begin{multicols*}{2}

\subsection*{\MakeUppercase{The story}}

\textbf{The Race}

Stacks of dry, cracked parchment sat next to dusty tomes on a table's blackened, rough-cut stone slab.
Catherine slumped over the numerous texts stashed throughout the necromancer's chambers while her eyes frantically scanned the pages.
She was searching for any clue to her enemy's plans.

After fruitless hours, she huffed and sent a moldering volume crashing through a stack of paper as exhaustion threatened to overcome her.
She was about to spin and leave the room when the corner of her eye caught a particular scribbling, revealed by her destructive act.
An ornate diagram, hastily scribbled, depicted what Catherine supposed might be some kind of necromantic summoning ritual.
Pulling the frail parchment closer, while most of it didn't make sense to her, it didn't take a mage to know what the dark drawings detailed.
They were the ritual instructions for raising Ghost Dragons from dead dragons.

Then it dawned on her.
Old tales told of an ancient Dragon Utopia in the badlands.
Vhidhax must be attempting to overtake the majestic creatures and use them against her in a last-ditch effort.
She frowned at the thought of her weakened enemy being able to defeat the dragons.
But if he did… If she missed something, he would have newfound strength enough to overpower her battered forces.
She clenched her fist, crushing the frail paper to dust before rushing to the doorway to gather who she could.

To her surprise, the necropolis' courtyard was abuzz with energy.
She placed her hands on the overlook's railing and peered over the edge to spy a host of knights in the color and flair of various baronies.

Rion came rushing up to her a moment later, intent on delivering news of their new guests.
``Queen Catherine!'' he said excitedly.
``Lordships around Steadwick heard of the tragedy and have sent their champions to aid us.
The knights spent days tracking and catching up to us, and they mean to join us and see the campaign through.''

Catherine smiled wide.
``Then tell them to get ready; the race is on.''

\textcolor{darkcandyapplered}{Add a Few Champions to your army.}

\textbf{Checkmate}

The broken remains of Vhidhax looked no grander than any of his many minions.
Necromantic energy had stripped the flesh from him long ago; without that power, his presence and ill-willed malevolence were gone as well.

Catherine wiped blood, sweat, and dirt from her face.
Rion had been right; the necromancer was not her father, but perhaps his slaying would ease the sting of her family discord and the sting of betrayal, at least a little.

Rion approached, and Catherine turned before embracing him in a rare moment of emotion.
The last time he'd seen his queen like this was when they were in Steadwick together, and he looked a little embarrassed as he was lost for words.
The two studied each other momentarily, silently checking on each other.
Rion thought she looked much older and weathered from the ordeal, but he was happy to see a twinkle in his queen's eyes again.

Her duty fulfilled and her seat secure, Catherine would see Steadwick rebuilt with Erathia just a little safer… for now.

\begin{itemize}
    \item \textcolor{darkcandyapplered}{If neither you or the enemy reached the Dragon Utopia, read ``A Queen's Right''}
    \item \textcolor{darkcandyapplered}{If the enemy army reached the Dragon Utopia before being defeated, read ``A Queen's Lament''}
    \item \textcolor{darkcandyapplered}{If you reached the Dragon Utopia and defeated the dragons within before defeating the enemy, read ``A Queen's Fury''}
\end{itemize}

\textbf{A Queen's Fury}

Catherine's hard eyes scanned the badlands from a window high on the Dragon Utopia.
While coming to blows with the famed dragon guardians was regrettable, she'd brought even them to heel, and the palace was hers.
No one would challenge her reign, not with the badlands under her control or the remaining dragons in her retinue.

Worn and tired faces looked up at her outside the fortress, like beggars asking for scraps.
What was left of who she'd brought into the hostile landscape was silently requesting the right to return home, yet the badlands were an enticing prize, and the bloodlust whipped up in the Queen of Erathia hadn't subsided yet.

It took much convincing from Rion before Catherine finally softened.
Reports from Steadwick were thrust in her face, and the mage's gentle nature reminded her that there were still those in her lands who needed her guidance.
So, finally, she ordered only a select few to watch over the badlands with the remaining mercenaries and dragons while she returned home with the rest.

\textbf{A Queen's Right}

A strange, deep calling swelled within Catherine.
She turned her focus to the majestic Dragon Utopia, with its rust-colored walls and pointed turrets.
It was an odd place for dragons to call home, yet its sheer presence was undeniable.

Catherine approached slowly, with the remains of her battered forces following cautiously.
Reaching the foot of the fortress, it was only a moment before the majestic snout of an azure dragon crawled into view from the large, curved, open entry to the stony fortress.
The rest of the creature soon followed, and with a stretch of wings, it stared down at the human queen.
While many fell to their knees, Catherine looked up to meet the dragon's gaze, and the two shared a moment of understanding between royalty.
It appeared the dragon knew what fate it had been spared from, and in thanks, it bowed to the human.
After a reflective moment, Catherine nodded, then turned, satisfied with the interaction.

His steed at a trot, Rion quickly approached.
``Well? Are we going inside?'' he asked curiously.

``There's no need,'' Catherine replied.
``I saw it in her eyes; the dragon knows what we did today.
We have their thanks and perhaps a favor to call on in the future.
Gods know we need it…''

Catherine drew her sword and reared her own steed high, presenting herself before her remaining forces with the dragon silhouetted behind her.
After a rousing speech that stirred the hearts of man and beast alike, she galloped to the rear lines to turn the formation toward home.

\textbf{A Queen's Lament}

Reality set in after the adrenaline and euphoria of a hard-fought victory wore off.
The battlefield was a nightmare of horrors.
Ghost Dragon corpses faded away as the last of their power died with them, and the scattered, broken bodies of both sides lay in clumps of gory remains.
The cost had been high…

Catherine regretted she couldn't save the dragons from their fate.
If things had been different, they might have been strong allies or at least sentinels to keep the local badland creatures in check.
Now, it was hard to tell what would leak from the unchecked wilds into other regions or townships under her charge.
Caution had won her the day, but she wondered if it was enough.
Would actions like this cause her downfall one day?

Home felt distant in the haze of fury that Catherine had been prisoner to.
As she trotted past the remains of those who had fought for her, her heart swelled with emotion.
It was war, and loss was inevitable.
She turned to the remaining few and threw a weak hand forward out of the badlands.
It was time to pick up the pieces.

\end{multicols*}



\addscenariogroup{\campaignstitle}{\layout/campaigns.png}

\clearpage

\input{\campaignspath/inferno_devilish_plan.tex}

\clearpage

% !TeX spellcheck = en_US
\addscenariosection[subsection]{1}{Inferno Campaign $-$ Dungeons and Devils}{2. Steadwick's Fall}{\images/firewall.png}

\begin{multicols*}{2}

\textbf{Author:} Tm335

\textbf{Source:} \href{https://discord.com/channels/740870068178649108/1246353361456861276/1246353361456861276}{Archon Studios Discord}

\textit{Catherine Ironfist has enlisted aid from Bracada and AvLee.
She knows we are close to Steadwick.
We must occupy Steadwick before she arrives.
Once we own Erathia's capitol, not even Catherine Ironfist will wrench it from our hands.}

\subsection*{\MakeUppercase{Scenario Length}}

This Scenario plays out over 16 Rounds.

\subsection*{\MakeUppercase{Player Setup}}

\textbf{Faction:} Inferno

\textbf{Faction Hero:} Choose any

\textbf{Starting Resources:} 15 \svg{gold}, 1 \svg{building_materials}, 1 \svg{valuables}

\textbf{Starting Income:} 10 \svg{gold}, 2 \svg{building_materials}, 0 \svg{valuables}

\textbf{Starting Units:}

\begin{itemize}
  \item A Few Troglodytes
  \item A Few Evil Eyes
  \item A Pack of Familiars
\end{itemize}

\textbf{Town Buildings:} \svgunit{bronze} Dwelling, \svgunit{silver} Dwelling

\textbf{Bonus:} Choose one of the following options:
\begin{itemize}
  \item Add a Pack of Harpies to your hand.
  \item Add a Pack of Magogs to your hand.
  \item Add an Ammo Cart to your hand.
  \item Search (4) the Spell Deck.
\end{itemize}

\subsection*{\MakeUppercase{AI Hero Setup}}

\textbf{Faction:} Castle

\textbf{Enemies:} General Kendal, Charging Heroes

\textbf{General Kendal's Army:} A Pack of Archangels, a Pack of Champions, a Pack of Zealots, a Pack of Crusaders, a Pack of Griffins, a Balista War Machine

\textbf{General Kendal's Deck:} 5 × Might Card, 1 × Magic Card, 3 × Skill Card

\textbf{General Kendal's Spell Deck:} 2 × Haste Spell Card

\textbf{General Kendal's Skill:} Artillery Ability Card\footnote{For General Kendal, the Artilery Ability Card always resolves the Expert effect and the Ballista War Machine Activates every time the Artillery Ability Card is drawn, as well as at the beginning of a Combat Round.}

\textbf{Charging Heroes' Factions:} Rampart, Tower, Castle

\textbf{Charging Heroes' Armies:} Neutral Army is one Level higher than your Hero Level (Max Level VI)\footnote{See page 35, ``Field Difficulty Level Table'' in the Core Rulebook, for further details on the number of Neutral Units you have to draw for this Neutral Army.}.

\textbf{Charging Heroes' Deck:} 2 × Might Card, 2 × Magic Card

\textbf{Charging Heroes' Spell Deck:} 2 × Slow Spell Card\footnote{All the Charging Heroes' Enemies use the same AI and Spell Decks. Reset them after every Combat.}.

\subsection*{\MakeUppercase{Map Setup}}

Take the following Map Tiles and set them up as shown in the Scenario map layout:

\textbf{2 × Starting (I) Map Tile}
\begin{itemize}
  \item 1 × Inferno (S6)
  \item 1 × Castle (S3)
\end{itemize}

\textbf{3 × Far (II--III) Map Tile}
\begin{itemize}
  \item 1 × Castle (F3)
  \item 1 × Rampart (F10)
  \item 1 × Dungeon (F2)
  \item 1 × Tower (\#F1)
  \item 1 × Rampart (choose from: F11, F12)
  \item 1 × Necropolis (choose from: F4, \#F6, F7)
\end{itemize}

\textbf{2 × Near (IV--V) Map Tile}
\begin{itemize}
  \item 2 × Castle (N3, choose from: \#N3, N6)
  \item 1 × Rampart (N8)
  \item 1 × Necropolis (N4)
\end{itemize}

\subsection*{\MakeUppercase{Heroes Placement}}

The Enemy Hero General Kendal is represented by one Castle Faction Hero model and appears on the center Field of the S3 Starting Map Tile.

The Castle Charging Hero is represented by one Castle Faction Hero model and appears on (and owns) the Settlement of the F3 Map Tile.

The Rampart Charging Hero is represented by one Rampart Faction Hero model and appears on (and owns) the Settlement of the F10 Map Tile.

The Tower Charging Hero is represented by one Tower Faction Hero model and appears on (and owns) the Settlement of the \#F1 Map Tile.

Place your Main Hero on the center Field of the Inferno Starting S6 Map Tile.

Place your Secondary Hero, represented by one Dungeon Faction Hero model of your choice, on the Settlement of the F2 Map Tile. This Settlement produces no resources.

\subsection*{\MakeUppercase{Victory Conditions}}

Defeat all 3 Enemy Controlled Settlements (F3, F10, \#F1), and capture Steadwick (S3) before
Queen Catherine Ironfist arrives at the end of the Round 16.

\subsection*{\MakeUppercase{Defeat Conditions}}

You lose one Combat encounter with your Main Hero (Surrendering costs 10 \svg{gold}, and does not count as a defeat).

You lose your Faction Town on the S6 Map Tile.

You run out of time -- you have time till the end of the Round 16.

\subsection*{\MakeUppercase{Timed Events}}

\textbf{\nth{1} Round:}
\begin{itemize}
  \item Read: ``You are to be congratulated on your progress so far.
    You have laid waste to Eastern Erathia, and are now within striking distance of the Erathian
    capital of Steadwick. You must capture the capital quickly!''
\end{itemize}

\textbf{\nth{2} Round:}
\begin{itemize}
  \item Read: ``Not only have Bracada and AvLee sent reinforcements, but we have received news that
    Queen Catherine Ironfist is marching a sizeable army from the south. We must control the capital and its
    garrisons before she arrives.

    You have just received a report on the progress of Queen Catherine.
    Forces from Nighon and Eeofol are attempting to delay her march to Steadwick,
    but doubt that they can delay her more than two or three months.''
\end{itemize}

\textbf{\nth{5} Round:}
\begin{itemize}
  \item Read: ``You receive a report from the south. Queen Catherine's forces have been sufficiently delayed,
    allowing you at least two months more to reach the capitol, but our own forces have suffered significant
    losses. Do not let their sacrifice go to waste.''
\end{itemize}

\textbf{\nth{10} Round:}
\begin{itemize}
  \item For any Enemy Settlements (F3, F10, \#F1) that have not been defeated, an additional Enemy
    Charging Hero of that Faction appears on each Settlement at the beginning of this Round.
\end{itemize}

\textbf{\nth{11} Round:}
\begin{itemize}
  \item Read: ``You receive a report from the south. Our forces continue to throw themselves in the path of
    Queen Catherine's armies, yet she continues to march northward. You have, at most, three or four weeks
    before she can reach the capital.''
\end{itemize}

\textbf{\nth{13} Round:}
\begin{itemize}
  \item Warning: ``Queen Catherine's march continues -- her forces are just two weeks away. If you do not hurry,
    we will not have time to secure the capital before her arrival.''
\end{itemize}

\textbf{\nth{15} Round:}
\begin{itemize}
  \item Warning: ``Your scouts report sighting Queen Catherine's army seven days to the southwest. If she
    reaches the capitol before you, all is lost.''
\end{itemize}

\textbf{\nth{16} Round:}
\begin{itemize}
  \item If Steadwick has not been taken, read: ``This morning, a massive army lead by Queen Catherine
    Ironfist arrived at the Erathian capitol of Steadwick. We have no choice but to retreat our forces.
    You have failed us... miserably.''
\end{itemize}

\textbf{When you complete the Scenario:}
\begin{itemize}
  \item Read: ``Congratulations! You captured Steadwick and are victorious!''
\end{itemize}

\subsection*{\MakeUppercase{Additional Rules}}

During this ``Inferno'' Campaign Scenario, the following rules apply:

\begin{itemize}
    \item The General Kendal Hero does not move and only waits in their Town. They start with Walls, a Gate,
      and an Arrow Tower on their side of the Combat Board.
    \item The difficulty Level of every Combat encounter on the map increases by one till the end of the Scenario
      (see page 35, ``Field Difficulty Level Table'' in the Core Rulebook).
    \item The Enemy Charging Heroes have only 2 Movement Points, instead of 3. They ignore everything else
      (including Mines and Settlements if possible) and go straight for the player's Faction Town (on Map Tile S6).
      They do not pursue the player directly, but if they happen to be on the same Map Tile, they will
      attack the player's Main Hero (not Secondary).
    \item The Enemy Charging Heroes move after the player's Turn ends.
    \item The borders on the Castle Starting Map Tile S3 cannot be crossed by any means until
      all Enemy Settlements (F3, F10, \#F1) have been defeated.
    \item Defeating Enemy Charging Heroes provides 2 \svg{valuables}.
    \item Obelisks give you 1 \svg{valuables} and function as a ``Castle
      Gate'' location (only if you have the ``Castle Gate'' built in your Town).
\end{itemize}

\vspace*{\fill}
\begin{center}
  \framedimage[0.9\linewidth]{\art/demon.jpg}
\end{center}
\vspace*{\fill}

\end{multicols*}

\begin{tikzpicture}[remember picture, overlay]
  \node(bg)[anchor=center, yshift=37em, opacity=0.17] at (current page.south) {
    \includegraphics[width=0.85\paperwidth, keepaspectratio]{\art/fire_shield.png}
  };
  \node(map)[anchor=center] at (current page.center) {
    \includegraphics[width=\textwidth]{\maps/inferno_steadwicks_fall.png}
  };

  \node at (3, -11.5) {\large{{\textbf{\textcolor{darkcandyapplered}{N3}}}}};
  \node at (4.4, -18.6) {\large{{\textbf{\textcolor{darkcandyapplered}{\#F1}}}}};
  \node at (15, -5.2) {\large{{\textbf{\textcolor{darkcandyapplered}{F10}}}}};
  \node at (17.3, -11.5) {\large{{\textbf{\textcolor{darkcandyapplered}{S6}}}}};
  \node at (8.8, -18.8) {\large{{\textbf{\textcolor{darkcandyapplered}{F3}}}}};
  \node at (15.9, -18.8) {\large{{\textbf{\textcolor{darkcandyapplered}{F2}}}}};
  \node at (2, -4.5) {\large{{\textbf{\textcolor{darkcandyapplered}{Steadwick}}}}};
  \node at (2, -5.2) {\large{{\textbf{\textcolor{darkcandyapplered}{S3}}}}};
  \node at (7.5, -5.4) {\large{{\textbf{\textcolor{darkcandyapplered}{Rampart}}}}};
  \node at (7.5, -6.1) {\large{{\textbf{\textcolor{darkcandyapplered}{Map Tiles}}}}};
  \node at (12.4, -17.8) {\large{{\textbf{\textcolor{darkcandyapplered}{Necropolis}}}}};
  \node at (12.4, -18.5) {\large{{\textbf{\textcolor{darkcandyapplered}{Map Tiles}}}}};
  \node at (2, -12.3) {\large{{\textbf{\textcolor{darkcandyapplered}{Castle}}}}};
  \node at (2, -13) {\large{{\textbf{\textcolor{darkcandyapplered}{Map Tiles}}}}};
\end{tikzpicture}


\clearpage

\input{\campaignspath/inferno_deal_with_the_devil.tex}

\clearpage

% !TeX spellcheck = en_US
\cleardoublepage\phantomsection\addcontentsline{toc}{section}{\protect\numberline{} {} {} {} {}The Queen's Gambit}
\addscenariosection[subsection]{1}{Castle Campaign $-$ The Queen's Gambit}{1. Greek Gift}{\images/catherine.png}

\begin{multicols*}{2}

\textbf{Author:} Tyson Heckert

\textbf{Source:} \href{https://travelledtales.com}{travelledtales.com}

\textit{Steadwick is free, but that doesn't mean it is safe.
Dark powers still roam Erathia, and one now moves its pieces into position.
While Catherine grapples with her responsibility — and her demons — she must be a bulwark against evil.}

\subsection*{\MakeUppercase{Scenario length}}

This Scenario plays out over 8 Rounds.

\subsection*{\MakeUppercase{Player setup}}

\textbf{Faction:} Castle

\textbf{Faction Hero:} Any Castle Hero. If you are not Catherine, you are in her entourage for this story.

\textbf{Starting Resources:} 10 \svg{gold}, 3 \svg{building_materials}, 1 \svg{valuables}

\textbf{Starting Income:} 10 \svg{gold}, 0 \svg{building_materials}, 0 \svg{valuables}

\textbf{Starting Units:}
\begin{itemize}
  \item A Few Halberdiers, a Few Marksmen
\end{itemize}

\textbf{Town Buildings:} \svgunit{bronze} Dwelling, Citadel

\subsection*{\MakeUppercase{AI Hero setup}}

\textbf{Faction:} Necropolis

\textbf{Enemy Army:} A Pack of Skeletons, a Pack of Zombies, a Few Dread Knights, a Few Ghost Dragons

\textbf{Enemy Deck:} 4 x Might Cards

\textbf{Special:} Place the Cloak of the Undead King IV and VI Cards aside

\subsection*{\MakeUppercase{Map setup}}

Take the following Map Tiles and arrange them as shown in the Scenario map layout:

\textbf{1 × Starting (I) Map Tile}
\begin{itemize}
    \item 1 x Castle (S3)
\end{itemize}

\textbf{3 × Far (II--III) Map Tile}
\begin{itemize}
    \item 3 x Castle (F3, F6, F9)
\end{itemize}

\textbf{2 × Near (IV--V) Map Tile}
\begin{itemize}
    \item 2 x Castle (N3, N6)
\end{itemize}

\subsection*{\MakeUppercase{Heroes placement}}

Place your Castle Hero on the center Field of the S3 Castle starting map Tile.

Place a Necropolis Hero on the center Field of the N3 Castle Tile.

\subsection*{\MakeUppercase{Victory Conditions}}

Defeat the enemy army.

\subsection*{\MakeUppercase{Defeat Conditions}}
\begin{itemize}
  \item You lose one combat encounter
  \item You run out of time at the end of the 8th Round
\end{itemize}
\end{multicols*}

\newpage

\begin{multicols}{2}

\subsection*{\MakeUppercase{Timed Events}}

\textbf{\nth{1} Round:}
\begin{itemize}
  \item Read ``Distant Threat'' section
\end{itemize}

\textbf{If entering the settlement on F3:}
\begin{itemize}
  \item Read ``Reverent Allies'' section
\end{itemize}

\textbf{Before combat with the enemy Hero:}
\begin{itemize}
  \item Read ``Bait and Switch'' section
\end{itemize}

\textbf{When you complete the Scenario:}
\begin{itemize}
  \item Read ``The Chase Begins'' section
\end{itemize}

\subsection*{\MakeUppercase{Additional rules}}

During this Castle campaign Scenario, the following rules apply:

\begin{itemize}
  \item The enemy army does not move
  \item Your Hero's maximum Level is 4
  \item The Glory of Erathia building is unavailable
  \item The Settlement on F3 does not have neutral enemies
\end{itemize}

\columnbreak

\includegraphics[width=\linewidth, keepaspectratio]{\art/griffin.png}

\end{multicols}

\begin{tikzpicture}[overlay]
  \node at (8.6, -5.3) {
    \includegraphics[width=\textwidth]{\_assets/maps/castle_greek_gift.png}
  };
  \node at (10, -2.0) {\large{{\textbf{\textcolor{darkcandyapplered}{F3}}}}};
\end{tikzpicture}

\newpage

\begin{multicols*}{2}

\subsection*{\MakeUppercase{The story}}

\textbf{Distant Threat}

Catherine Ironfist rubbed her tired, worn thumb across the image of her father, King Gryphonheart, in a locket.
She wanted to remember the good, valiant parts of his life rather than his actions in unlife.
Her gaze turned to her cracked skin and mangled thumbnail, and it reminded her of how hard she had fought to be where she was.

The weight of Erathia rested on her shoulders.
It had only been weeks since the siege and recapture of Steadwick, and resources, as well as manpower, were in short supply.
Still, with time, the stony seat of the castle would be sufficient to further liberate the realm.

One such problem was weighing on her mind.
After replacing the locket in her hand with a missive, she re-read a particular entry about a rising necromancer named Vhidhax.
Rumors were that the fiend had stolen precious artifacts that further enhanced his power over the dead.
While rumors are usually best left to fishwives and children, this one aligned with information about an army growing in strength, and she dreaded the thought of the two problems colliding and crashing over her weary crown.

Late that evening, Catherine was venting to Rion about the relative drudgery of rebuilding Steadwick.
She'd just about fallen asleep in her chair when a heavy knock turned her attention to the room's large door that sported intricately carved shields from various long-gone baronies.

The wooden portal flung open before she could spring up, revealing a wide-eyed and panicked guard.
No words came from the man; whatever haunted him seemed absolute.
Instead, he merely pointed with a trembling hand toward the window across the room.

Catherine shot a look at Rion before the two darted toward the open air of the rectangular window.
What they saw shocked them, too, and while Catherine's heart sank, she gripped Rion's shoulder with surprising strength.
When he looked back at her with a slight wince, she only returned a stony look of resolve.

Danger had come to Steadwick again.
Atop a hill in the far distance, overlooking many of the defenseless farmsteads Catherine had in her charge, a predator swept its gaze like a wolf might hungrily eye sheep.
A magnificently terrifying azure hue wafted from a Ghost Dragon's massive, extended wings like an ethereal heat haze.

\textbf{Reverent Allies}

While Catherine trusted her horse, having fought tooth and nail with the battle-hardened steed for as long as she could remember, it shuddered in hesitation as they approached the majestic, columned main building of the settlement.
Catherine had also felt the almost unsettling holy power in the area when, suddenly, the sound of rushing wings caught her attention.
Her gaze snapped upward to catch sight of something approaching.

The early morning sun was bright, and the figure that swooped low was in front of it.
The silhouette of wings was visible, and Catherine thought it to be a Gryphon at first.
However, as it closed in on them, it became apparent it was the shape of a man with wings instead.
It caught Catherine's horse off guard, and she had to catch the reins and wrench them hard to steady herself and calm the frightened animal before it could fling her from her seat.

With lightning speed, the figure landed mere feet from them to reveal the stubbled, handsome face of a young man with golden, curled locs and impressively robust bronze armor sculpted to his muscular form.
He was one of the famed Archangels to have stayed near Steadwick after Catherine's victory.
While the forces behind her fell to their knees in reverence, she only nodded with stern resolve.

With a deeper, and more reverberating voice than any man, the angel was first to speak.
``Queen Catherine.
I expected you might come when I spied the rot taking form in the hills.''

Catherine responded with a smile.
It was good to see she still had allies; her forces would need bolstering to take on the undead.
``It's good to see you, my friend,'' she said.
``It seems Steadwick is threatened again, and I ride to root out the evil before it can seed.
Would you join me at my side?''

``Of course, my queen!'' the angel barked.
``Combined, no evil can thwart our advance!'' As the final words left his lips and began to echo, he drew his weapon.
The longsword ignited as it left its sheath, and its blazing glory had most of those kneeling averting their eyes.
Immediately after, he took to the air and began circling the area again.
Catherine smiled wide as she watched more angels appear, then looked back to her forces and extended an open palmed arm forward in a command to march.

\textcolor{darkcandyapplered}{Add a Few Archangels to your army.}

\textbf{Bait And Switch}

Catherine couldn't wait to drive her sword through the heart of the Ghost Dragon, then lop the head from the one commanding it.
She pushed her powerful mount forward, charging toward the waiting, taunting foes.
While the great number of shambling undead, mostly animated citizens from nearby townships, was impossible to ignore, they wouldn't stop her.
With a thunderous crash, she slammed past the forward lines, knocking off limbs and cracking bones.

Seconds later, she was halfway through the mass of bodies when the stench of rotten flesh and the true immensity of her enemy's numbers hit her.
Dread Knights flanked the dragon she'd been racing towards.
Worse still, a robed, decrepit-looking figure was watching from the rear of the enemy lines.
Despite its wretched appearance, it had an immensely powerful presence to it.
A stinging tingle shot up Catherine's spine as she felt, more than saw, its eyes return her gaze.
More than she could take on alone, she swung her steed in a circle to reform with her mounted allies, then looked to the sky just as an Archangel came crashing down onto the dragon.

Shockingly, instead of slamming into the beast, the angel flew through it like it was made of air.
He hit the ground with a crunching thud before, suddenly, the dragon's image began to shift and shimmer, then disintegrate.
The Dread Knights did the same, and replacing them were throngs of further undead that now threatened to encircle Catherine and her forward forces.

Catherine's eyes shot toward what she believed was the necromancer.
It stood motionless a moment longer before cackling laughter cut through the air, and its image wafted away into the night air just as the others had.

It was a trap.
The necromancer and his strongest forces were not there, and Catherine would have to fight her way out from under a crush of undead, wasting precious time and lives.

\begin{itemize}
  \item \textcolor{darkcandyapplered}{Replace the Ghost Dragons and Dread Knights with \svgunit{bronze} neutral Zombies and Skeletons.}
  \item \textcolor{darkcandyapplered}{Play ``Cloak of the Undead King IV'' and ``Cloak of the Undead King VI'' onto the packs of Zombies and Skeletons, respectively.}
  \item \textcolor{darkcandyapplered}{Remove the AI Might and Magic Deck from the game, the undead fight on their own.}
\end{itemize}


\textbf{The Chase Begins}

Catherine huffed, sweat beading on her forehead as the last undead fell with wet thuds.
Her forces were strong, and simple reanimations weren't the kind of threat to stop them.
She just hoped… Her stomach churned as her gaze turned back to Steadwick.

An orange glow, far too bright for torchlight, was coming from the castle.
As she watched in horror, thick black smoke began to rise from the towers and parapets.
Steadwick was burning.

Rion quickly caught up to her, his horse panting as hard as he was.
He began to speak but couldn't find the words as his mind raced.
Instead, he placed a hand on his queen's shoulder.

The necromancer had played his opening move and won.

Just before Catherine turned to her regrouping forces, she spotted the spectral blue of the real Ghost Dragon in the distance.
Its work destroying Steadwick's defenses had been finished.

Archangels circled overhead, and Catherine threw her head to the sky, tears welling at the thought of losing what she had worked so hard to build.
``Find him!'' she roared, addressing whatever gods were listening as well as the angels above.

``Is that… wise?'' Rion asked softly.

Catherine responded with fire in her eyes.
``What good is rebuilding a ruin when the fiend will just continue to play his tricks?'' she said.
``No.
If we want to save Steadwick, we have to end him before he can further strengthen himself.''

\begin{itemize}
  \item \textcolor{darkcandyapplered}{Remove Archangels from your army.}
  \item \textcolor{darkcandyapplered}{Keep the rest of your forces and resources for the next Scenario.}
  \item \textcolor{darkcandyapplered}{Reset your Hero per the normal campaign rules.}
\end{itemize}

\columnbreak

\begin{center}
  \framedimage[0.8\linewidth]{\art/ghost_dragon.jpg}
\end{center}

\end{multicols*}


\clearpage

% !TeX spellcheck = en_US
\addscenariosection[subsection]{1}{Castle Campaign $-$ The Queen's Gambit}{2. Sicilian Dragon}{\images/catherine.png}

\begin{multicols*}{2}

\textbf{Author:} Tyson Heckert

\textbf{Source:} \href{https://travelledtales.com}{travelledtales.com}

\textit{Convinced that the only way to save Steadwick is to attack Vhidhax, Catherine powers her small force ahead.
Following the Archangels to a distant, inhospitable land, she must now navigate her way through.}

\subsection*{\MakeUppercase{Scenario length}}

This Scenario plays out over 11 Rounds.

\subsection*{\MakeUppercase{Player setup}}

\textbf{Faction:} Castle

\textbf{Faction Hero:} The same Hero as Scenario 1

\textbf{Starting Resources:} The resources you ended with in Scenario 1

\textbf{Starting Income:} 10 \svg{gold}, 0 \svg{building_materials}, 0 \svg{valuables}

\textbf{Starting Units:} The army you ended with in Scenario 1

\textbf{Town Buildings:} None

\textbf{Bonus:} Begin the Scenario at Hero Level 2, but do not take new Ability Cards.

\subsection*{\MakeUppercase{AI Hero setup}}

\textbf{Faction:} Necropolis

\textbf{Enemy Army:} A Pack of Skeletons, a Pack of Zombies, a Pack of Liches, a Few Dread Knights, a Few Ghost Dragons

\textbf{Enemy Deck:} 3 x Might Cards, 3 x Magic Cards, 1 x Skill Card

\textbf{Enemy Spells:} 2 x Lightning Bolt, 1 x Curse

\textbf{Enemy Skills:} Cloak of the Undead King IV and VI, the enemy will play whichever it can

\textbf{Neutral Enemy Deck:} 4 x Might Cards

\subsection*{\MakeUppercase{Map setup}}

Take the following Map Tiles and arrange them as shown in the Scenario map layout:

\textbf{1 × Starting (I) Map Tile}
\begin{itemize}
    \item 1 x Necropolis (S1)
\end{itemize}

\textbf{3 × Far (II--III) Map Tile}
\begin{itemize}
    \item 3 x random Tiles from Necropolis or Dungeon (F1, F2, F4, F5, F7, F8)
\end{itemize}

\textbf{4 × Near (IV--V) Map Tile}
\begin{itemize}
    \item 2 x Necropolis (N1, N4)
    \item 2 x Dungeon (N2, N5)
\end{itemize}

\textbf{1 × Center (VI--VII) Map Tile}
\begin{itemize}
  \item 1 × Grail Tile (C2)
\end{itemize}

\textbf{\MakeUppercase{Note:}} Place Tile S1 aside, face up.

\subsection*{\MakeUppercase{Heroes placement}}

Place your Castle Hero on the empty Field of the rightmost Far Tile.

Place a Necropolis Hero on the center Field of the S1 Necropolis Tile.

\subsection*{\MakeUppercase{Victory Conditions}}

\begin{itemize}
  \item Find the Archangels
  \item Defeat the enemy army
\end{itemize}


\subsection*{\MakeUppercase{Defeat Conditions}}
\begin{itemize}
  \item You lose one combat encounter
  \item You run out of time at the end of the 11th Round
\end{itemize}

\end{multicols*}

\begin{multicols}{2}

\subsection*{\MakeUppercase{Timed Events}}

\textbf{\nth{1} Round:}
\begin{itemize}
  \item Read ``Into the Badlands'' section
\end{itemize}

\textbf{Visiting the first Obelisk:}
\begin{itemize}
  \item Read ``The First Relic'' section
\end{itemize}

\textbf{Visiting the second Obelisk:}
\begin{itemize}
  \item Read ``Showdown'' section
\end{itemize}

\textbf{When you complete the Scenario:}
\begin{itemize}
  \item Read ``The Escape'' section
\end{itemize}

\subsection*{\MakeUppercase{Additional rules}}

During this Castle campaign Scenario, the following rules apply:

\begin{itemize}
  \item The enemy army does not move
  \item Your Hero's maximum Level is 6
  \item All buildings are unavailable, building is not possible
\end{itemize}

\columnbreak

\begin{itemize}
\item All neutral enemies fight using the Neutral Enemy Deck.
\item Each time you visit a Trading Post, draw the top Cards from the \svgunit{bronze} bronze, \svgunit{silver} silver, and \svgunit{golden} golden neutral enemy Decks.
You may recruit any, or all of them by paying their cost.
Shuffle any un-hired ones back into their Decks.

\item The Grail may be sold at a trading post for 30 \svgunit{gold} \textit{or} \textbf{Search (3)} the relic Deck, twice.

\item When fighting the enemy at thier Town, they have a Citadel.
Don't forget to add the walls and arrow tower to make the battle a siege.
\end{itemize}

\end{multicols}

\begin{tikzpicture}[overlay]
  \node at (9.0, -6.0) {
    \includegraphics[width=0.80\paperwidth]{\_assets/maps/castle_sicilian_dragon.png}
  };
\end{tikzpicture}

\newpage

\begin{multicols*}{2}

\subsection*{\MakeUppercase{The story}}

\textbf{Into the Badlands}

Catherine crossed arms and leaned on her saddle horn as she peered out over the wasteland they'd been led to.
The Archangels had gone ahead of the small force she'd scraped together, and she wondered if it had been a mistake to let them go.
With the enemy stronghold still nowhere in sight, locating it would take time, and the inhospitable landscape would not be kind.

Rion pulled up next to her as he finished gulping the last contents of a water pouch.
He wiped the dribble and sweat from his face before saying anything.
``Not the best place to do any fighting.
Besides the poor conditions, I've heard the locals are known for their particular… ferocity.''

``Well then, we'll just have to bring them to heel,'' Catherine replied coldly.
The truth was she was more worried about that possibility than she let on, but she had to maintain her composure to protect the image of her strength.

``You think they'll fight for us?'' Rion asked with a raised eyebrow.

Catherine tightly gripped a heavy sack of coin by her side and heaved it into her lap so the contents jingled like tiny bells to emphasize her point.
``I think generous arrangements can be made.
For the rest, the sword will do.''

\textbf{The First Relic}

Ahead, a clearing of dry, cracked ground gave way to a large Obelisk protruding from the ground like a massive marker.
Catherine had seen enough magical artifacts in her time that she could tell the structure was a place of power.
Throwing an arm forward, she signaled Rion to proceed with an inspection.

The mage approached slowly, removing a glove to touch the cold stone of the smoothly cut sides with his skin.
Intricate runes were carved in it, detailing all manner of scripture in long-lost tongues.
As awe-inspiring as it was, equally apparent was the time-worn damage.

After a moment, he turned to Catherine.
``It's a transport beacon,'' he said.
``Like a monolith, but with a shorter range.
At least… it was.
It's damaged, and where the magic in the runes should be strong, even glowing, they're dark, and any power is long decayed.''

Catherine dismounted and slowly inspected the area before saying anything.
``Some of the damage is recent.
I can still feel residual heat; the Archangels were here.''

``How can you be sure? Surely, some would have found us by now and notified us of their progress.''

``They've never let me down,'' Catherine grunted.
``Come, let's see if other such beacons exist.''

\textbf{Showdown}

Sounds of battle perked the ears of Catherine's forces, and she urged her steed forward to catch sight of the combatants.

Rounding crags and jagged rocks jutting at all angles, another Obelisk revealed itself with the final moments of a battle taking place around it.
Several Archangels were on the ground, locked in combat with the remains of a larger undead force.

Approaching at speed, Catherine raised a fist in the air to salute the holy warriors, and several rasped their bronze chestplates in return.

``Queen Catherine, you've saved us the trouble of finding you,'' the lead angel said in his deep voice.
He gestured to the Obelisk with an open palm as the sounds of battle finally died.
``The foes meant to keep us from these structures by destroying them.
We barely managed to intercept them in time to preserve this one.''

Rion trotted past them as they spoke, intent on inspecting the intact Obelisk.
As before, he began to place his hand on the cold stone, but recoiled before making contact.
``It's active!'' he said excitedly.
``This may be the break we need.
With luck, this will take us directly to the enemy stronghold.''

Catherine smiled, reaching to clasp the arm of the angel in thanks for his work.
She looked back at the ragtag group of forces she'd gathered and hoped it would be enough for a siege.

\begin{itemize}
  \item \textcolor{darkcandyapplered}{Add a Few Archangels to your army.}
  \item \textcolor{darkcandyapplered}{You may now teleport to the enemy town on Tile S1.
  If you choose not to, you may in the future by spending one movement at this Obelisk.}
\end{itemize}


\textbf{The Escape}

With the thick, macabre, almost demonic-looking walls breached and the bulk of the necromancer's forces slain, the castle belonged to Catherine.
It was an eye for an eye, and she relished driving the final blow into the last enemy ranks still in the courtyard herself.

Unfortunately, after a frantic search of the grounds, Vhidhax himself was nowhere in sight.
Catherine raced up the walls just in time to peer over the side and catch sight of the necromancer fleeing with a sizable retinue that had quietly slipped away during the battle.
He was traveling north, further into the badlands and away from her home.
Catherine's campaign wasn't finished yet.

Catherine turned to observe the battlefield from above.
The butcher's bill had been large.
Both fresh and long-decaying bodies lay strewn about, several Archangels among them.
Rion caught up with her a minute later, and the two exchanged looks of exhaustion.

``It's done,'' Rion said through panting breaths.

``No.
The necromancer lives, and he flees north.''

``But… our forces.
The Archangels are too few to continue; what hope do we have?''

``There is always hope, my friend.
We'll use this castle as a base to resupply, and then we need to give chase before the fiend regains his strength.
Steadwick will never be safe otherwise!''

``This is foolish,'' Rion sighed.
``We don't need to fight this hard; Vhidhax is not your father, Catherine…''

\begin{itemize}
  \item \textcolor{darkcandyapplered}{Remove Archangels from your army, if possible.}
  \item \textcolor{darkcandyapplered}{Keep the rest of your army and resources for the next Scenario.}
  \item \textcolor{darkcandyapplered}{Reset your Hero per the normal campaign rules.}
\end{itemize}

\vspace{2em}
\begin{center}
  \framedimage[0.75\linewidth]{\art/necropolis_town.jpg}
\end{center}

\end{multicols*}


\clearpage

% !TeX spellcheck = en_US
\addscenariosection[subsection]{1}{Castle Campaign $-$ The Queen's Gambit}{3. Two Knights Defense}{\images/catherine.png}

\begin{multicols*}{2}

\textbf{Author:} Tyson Heckert

\textbf{Source:} \href{https://travelledtales.com}{travelledtales.com}

\textit{With Vhidhax's stronghold in her grasp, Catherine needs to discover and halt the necromancer's next move.}

\subsection*{\MakeUppercase{Scenario length}}

This Scenario plays out over 8 Rounds.

\subsection*{\MakeUppercase{Player setup}}

\textbf{Faction:} Castle

\textbf{Faction Hero:} The same Hero you played in Scenario 2

\textbf{Starting Resources:} The resources you ended Scenario 2 with, plus 10 \svg{gold}

\textbf{Starting Income:} 10 \svg{gold}, 0 \svg{building_materials}, 0 \svg{valuables}

\textbf{Starting Units:} The army you ended Scenario 2 with

\textbf{Town Buildings:} \svgunit{bronze} Dwelling, \svgunit{silver} Dwelling, Citadel

\textbf{Bonus:} Begin the Scenario at Hero Level 3, but do not take new Ability Cards.

\subsection*{\MakeUppercase{AI Hero setup}}

\textbf{Faction:} Necropolis

\textbf{Enemy Army:} A Pack of Skeletons, a Pack of Zombies, a Pack of Liches, a Pack of Dread Knights

\textbf{Enemy Deck:} 3 x Might Cards, 3 x Magic Cards, 1 x Skill Card

\textbf{Enemy Spells:} 2 x Lightning Bolt, 1 x Curse

\textbf{Enemy Skills:} Cloak of the Undead King IV and VI, the enemy will play whichever it can

\subsection*{\MakeUppercase{Map setup}}

Take the following Map Tiles and arrange them as shown in the Scenario map layout:

\textbf{1 × Starting (I) Map Tile}
\begin{itemize}
    \item 1 x Necropolis (S1)
\end{itemize}

\textbf{2 × Far (II--III) Map Tile}
\begin{itemize}
    \item 2 x random Tiles from Necropolis or Dungeon (F1, F2, F4, F5, F7, F8)
\end{itemize}

\textbf{4 × Near (IV--V) Map Tile}
\begin{itemize}
    \item 2 x Necropolis (N1, N4)
    \item 2 x Dungeon (N2, N5)
\end{itemize}

\textbf{1 × Center (VI--VII) Map Tile}
\begin{itemize}
  \item 1 × Dragon Utopia Tile (C1)
\end{itemize}

\subsection*{\MakeUppercase{Heroes placement}}

Place your Castle Hero on the center Field of the Necropolis start Tile S1.

Place a Necropolis Hero on the bottom left Field of the leftmost Far Tile.
Reveal the Tile when the game begins.

\subsection*{\MakeUppercase{Victory Conditions}}

\begin{itemize}
  \item Defeat the enemy army
\end{itemize}


\subsection*{\MakeUppercase{Defeat Conditions}}
\begin{itemize}
  \item You lose one combat encounter
  \item You run out of time at the end of the 8th Round
\end{itemize}
\end{multicols*}

\begin{multicols}{2}

\subsection*{\MakeUppercase{Timed Events}}

\textbf{\nth{1} Round:}
\begin{itemize}
  \item Read ``The Race'' section
\end{itemize}

\textbf{When you complete the Scenario:}
\begin{itemize}
  \item Read ``Checkmate'' section
\end{itemize}

\subsection*{\MakeUppercase{Additional rules}}

During this Castle campaign Scenario, the following rules apply:

\begin{itemize}
  \item The Glory of Erathia building is unavailable.
  \item Each time you visit a Trading Post, draw the top Cards from the \svgunit{bronze} bronze, \svgunit{silver} silver, and \svgunit{golden} golden neutral enemy Decks.
  You may recruit any, or all of them by paying their cost.
  Shuffle any un-hired ones back into their Decks.

  \item After defeating the neutral army at the Dragon Utopia, add a random dragon from the \svgunit{azure} Azure neutral Deck to your army.
  \item The enemy army follows the normal AI movement rules, stopping to capture Settlements and Mines, with its ultimate goal to reach the Dragon Utopia.
  \item If the enemy army reaches the Dragon Utopia before you do, flip the Pack side of Dread Knights to Few and add a Pack of Ghost Dragons.
\end{itemize}

\begin{tikzpicture}[overlay]
    \node at (-1.0, -8.5) {
      \includegraphics[width=0.70\paperwidth]{\_assets/maps/castle_two_knights_defense.png}
    };
  \end{tikzpicture}
\end{multicols}

\newpage

\begin{multicols*}{2}

\subsection*{\MakeUppercase{The story}}

\textbf{The Race}

Stacks of dry, cracked parchment sat next to dusty tomes on a table's blackened, rough-cut stone slab.
Catherine slumped over the numerous texts stashed throughout the necromancer's chambers while her eyes frantically scanned the pages.
She was searching for any clue to her enemy's plans.

After fruitless hours, she huffed and sent a moldering volume crashing through a stack of paper as exhaustion threatened to overcome her.
She was about to spin and leave the room when the corner of her eye caught a particular scribbling, revealed by her destructive act.
An ornate diagram, hastily scribbled, depicted what Catherine supposed might be some kind of necromantic summoning ritual.
Pulling the frail parchment closer, while most of it didn't make sense to her, it didn't take a mage to know what the dark drawings detailed.
They were the ritual instructions for raising Ghost Dragons from dead dragons.

Then it dawned on her.
Old tales told of an ancient Dragon Utopia in the badlands.
Vhidhax must be attempting to overtake the majestic creatures and use them against her in a last-ditch effort.
She frowned at the thought of her weakened enemy being able to defeat the dragons.
But if he did… If she missed something, he would have newfound strength enough to overpower her battered forces.
She clenched her fist, crushing the frail paper to dust before rushing to the doorway to gather who she could.

To her surprise, the necropolis' courtyard was abuzz with energy.
She placed her hands on the overlook's railing and peered over the edge to spy a host of knights in the color and flair of various baronies.

Rion came rushing up to her a moment later, intent on delivering news of their new guests.
``Queen Catherine!'' he said excitedly.
``Lordships around Steadwick heard of the tragedy and have sent their champions to aid us.
The knights spent days tracking and catching up to us, and they mean to join us and see the campaign through.''

Catherine smiled wide.
``Then tell them to get ready; the race is on.''

\textcolor{darkcandyapplered}{Add a Few Champions to your army.}

\textbf{Checkmate}

The broken remains of Vhidhax looked no grander than any of his many minions.
Necromantic energy had stripped the flesh from him long ago; without that power, his presence and ill-willed malevolence were gone as well.

Catherine wiped blood, sweat, and dirt from her face.
Rion had been right; the necromancer was not her father, but perhaps his slaying would ease the sting of her family discord and the sting of betrayal, at least a little.

Rion approached, and Catherine turned before embracing him in a rare moment of emotion.
The last time he'd seen his queen like this was when they were in Steadwick together, and he looked a little embarrassed as he was lost for words.
The two studied each other momentarily, silently checking on each other.
Rion thought she looked much older and weathered from the ordeal, but he was happy to see a twinkle in his queen's eyes again.

Her duty fulfilled and her seat secure, Catherine would see Steadwick rebuilt with Erathia just a little safer… for now.

\begin{itemize}
    \item \textcolor{darkcandyapplered}{If neither you or the enemy reached the Dragon Utopia, read ``A Queen's Right''}
    \item \textcolor{darkcandyapplered}{If the enemy army reached the Dragon Utopia before being defeated, read ``A Queen's Lament''}
    \item \textcolor{darkcandyapplered}{If you reached the Dragon Utopia and defeated the dragons within before defeating the enemy, read ``A Queen's Fury''}
\end{itemize}

\textbf{A Queen's Fury}

Catherine's hard eyes scanned the badlands from a window high on the Dragon Utopia.
While coming to blows with the famed dragon guardians was regrettable, she'd brought even them to heel, and the palace was hers.
No one would challenge her reign, not with the badlands under her control or the remaining dragons in her retinue.

Worn and tired faces looked up at her outside the fortress, like beggars asking for scraps.
What was left of who she'd brought into the hostile landscape was silently requesting the right to return home, yet the badlands were an enticing prize, and the bloodlust whipped up in the Queen of Erathia hadn't subsided yet.

It took much convincing from Rion before Catherine finally softened.
Reports from Steadwick were thrust in her face, and the mage's gentle nature reminded her that there were still those in her lands who needed her guidance.
So, finally, she ordered only a select few to watch over the badlands with the remaining mercenaries and dragons while she returned home with the rest.

\textbf{A Queen's Right}

A strange, deep calling swelled within Catherine.
She turned her focus to the majestic Dragon Utopia, with its rust-colored walls and pointed turrets.
It was an odd place for dragons to call home, yet its sheer presence was undeniable.

Catherine approached slowly, with the remains of her battered forces following cautiously.
Reaching the foot of the fortress, it was only a moment before the majestic snout of an azure dragon crawled into view from the large, curved, open entry to the stony fortress.
The rest of the creature soon followed, and with a stretch of wings, it stared down at the human queen.
While many fell to their knees, Catherine looked up to meet the dragon's gaze, and the two shared a moment of understanding between royalty.
It appeared the dragon knew what fate it had been spared from, and in thanks, it bowed to the human.
After a reflective moment, Catherine nodded, then turned, satisfied with the interaction.

His steed at a trot, Rion quickly approached.
``Well? Are we going inside?'' he asked curiously.

``There's no need,'' Catherine replied.
``I saw it in her eyes; the dragon knows what we did today.
We have their thanks and perhaps a favor to call on in the future.
Gods know we need it…''

Catherine drew her sword and reared her own steed high, presenting herself before her remaining forces with the dragon silhouetted behind her.
After a rousing speech that stirred the hearts of man and beast alike, she galloped to the rear lines to turn the formation toward home.

\textbf{A Queen's Lament}

Reality set in after the adrenaline and euphoria of a hard-fought victory wore off.
The battlefield was a nightmare of horrors.
Ghost Dragon corpses faded away as the last of their power died with them, and the scattered, broken bodies of both sides lay in clumps of gory remains.
The cost had been high…

Catherine regretted she couldn't save the dragons from their fate.
If things had been different, they might have been strong allies or at least sentinels to keep the local badland creatures in check.
Now, it was hard to tell what would leak from the unchecked wilds into other regions or townships under her charge.
Caution had won her the day, but she wondered if it was enough.
Would actions like this cause her downfall one day?

Home felt distant in the haze of fury that Catherine had been prisoner to.
As she trotted past the remains of those who had fought for her, her heart swelled with emotion.
It was war, and loss was inevitable.
She turned to the remaining few and threw a weak hand forward out of the badlands.
It was time to pick up the pieces.

\end{multicols*}



\include{sections/back_cover.tex}

\end{document}
